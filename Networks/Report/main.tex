\documentclass[a4page]{article}
\usepackage[14pt]{extsizes} % для того чтобы задать нестандартный 14-ый размер шрифта
\usepackage[utf8]{inputenc}
\usepackage[T1, T2A]{fontenc}
\usepackage[utf8]{inputenc}
\usepackage[russian]{babel} % поддержка русского языка
\usepackage{amsmath}  %  математические символы
\usepackage[left=20mm, top=15mm, right=15mm, bottom=30mm, footskip=15mm]{geometry} % настройки полей документа
\usepackage{indentfirst} % по умалчанию убирается отступ у первого абзаца в секции, это отменяет это.
\usepackage{paralist} % добавить компактные списки (compactitem, compactenum, compactdesc)

\usepackage{fancyvrb}
\usepackage{framed}
\usepackage{url}
\usepackage{csquotes}

\usepackage{float}
\floatstyle{ruled}

\usepackage[
	backend=biber,
	sorting=nyt,
	bibstyle=gost-numeric,
	citestyle=gost-numeric
]{biblatex}

\usepackage[
	bookmarks=true, colorlinks=true, unicode=true,
	urlcolor=black,linkcolor=black, anchorcolor=black,
	citecolor=black, menucolor=black, filecolor=black,
]{hyperref}

\addbibresource{sources.bib}

\renewcommand{\baselinestretch}{1.35}

\begin{document}

\begin{titlepage}
	\begin{center}
		\hfill \break
		\textbf{
			\large{РОССИЙСКИЙ УНИВЕРСИТЕТ ДРУЖБЫ НАРОДОВ}\\
			\normalsize{Факультет физико-математических и естественных наук}\\
			\normalsize{Кафедра прикладной информатики и теории вероятностей}\\
		}
		\vspace*{\fill}
		\Large{\textbf{Архитектура REST взаимодействия компонентов распределённого приложения в сети.\\Обозначение объектов JavaScript (JSON).}}
		\\
		\underline{\textit{\normalsize{Дисциплина: Вычислительные системы, сети и телекоммуникации}}}
		\vspace*{\fill}

	\end{center}

	\begin{flushright}
		Студент: \underline{Матюхин Григорий}\\ \vspace{0.5cm}
		Группа: \underline{НПИбд-01-21}
	\end{flushright}

	\begin{center} \textbf{МОСКВА} \\ 2023 г. \end{center}
	\thispagestyle{empty}

\end{titlepage}

\newpage

\tableofcontents

\newpage
\section{Введение}

\newpage
\section{REST}
REST (от англ. Representational State Transfer ---
<<передача репрезентативного состояния>> или <<передача "самоописываемого" состояния>>) ---
архитектурный стиль взаимодействия компонентов распределённого приложения в сети \cite{REST}.
Другими словами, REST --- это набор правил того,
как программисту организовать написание кода серверного приложения,
чтобы все системы легко обменивались данными и приложение можно было масштабировать.
REST представляет собой согласованный набор ограничений,
учитываемых при проектировании распределённой гипермедиа-системы.

Для веб-служб, построенных с учётом REST (то есть не нарушающих накладываемых им ограничений),
применяют термин <<RESTful>>.

\subsection{Свойства архитектуры REST}

\subsection{Требования к архитектуре REST}
\subsubsection{Модель клиент-сервер}
\subsubsection{Отсутствие состояния}
\subsubsection{Кэширование}
\subsubsection{Единообразие интерфейса}
\subsubsection{Слои}
\subsubsection{Код по требованию}

\subsection{Преимущества}

\subsection{Недостатки}

\newpage
\section{JSON}
JSON (англ. JavaScript Object Notation) ---
текстовый формат обмена данными, основанный на JavaScript \cite{rfc8259}\cite{ISO21778}.
Как и многие другие текстовые форматы, JSON легко читается людьми.
Формат JSON был разработан Дугласом Крокфордом.

Несмотря на происхождение от JavaScript (точнее, от подмножества языка стандарта ECMA-262 1999 \cite{ecma:262} года),
формат считается независимым от языка и может использоваться практически с любым языком программирования.
Для многих языков существует готовый код для создания и обработки данных в формате JSON.

\subsection{Использование}

\subsection{Синтаксис}

\subsection{YAML}

\newpage
\section{Заключение}

\newpage
\addcontentsline{toc}{section}{Список литературы}
\printbibliography
\end{document}

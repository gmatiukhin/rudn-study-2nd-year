\documentclass[12pt]{article}
\usepackage[utf8]{inputenc}
\usepackage[russian]{babel}
\usepackage{hyperref}
\usepackage{datetime}
\usepackage{amsmath}
\usepackage{amssymb}
\usepackage{amsfonts}
\usepackage{tikz}
\graphicspath{ {./Images/} }

% For slope Fields
\usepackage{pgfplots}
\usetikzlibrary{calc}
\usetikzlibrary{shapes.geometric}
\pgfplotsset{compat=1.8}

\author{Григорий Матюхин}
\date{\today}
\title{
	Теория вероятностей и математическая статистика \\
	\large Подготовка к экзамену.
}

\begin{document}
\maketitle
\newpage
\tableofcontents
\newpage

\section{Тема 1. Случайный эксперимент и пространство элементарных исходов}

\subsection{Случайный эксперимент}

\subsection{Условия проведения эксперимента}

\subsection{Элементарный исход эксперимента}
<<Неделимый>> исход эксперимента. $\omega$.

\subsection{Пространство элементарных исходов}
Все возможные исходы эксперимента. $\Omega = \{\omega_1, \omega_2, ..., \omega_n\}$.

\subsection{Наблюдаемые события}

\subsection{Модель эксперимента}
Совокупность объектов:
\begin{itemize}
	\item $\langle \mathcal{K}, \Omega, 2^{|\Omega|}\rangle$,
	      если все элементарные события $\omega \in \Omega$ являются наблюдаемыми.
	\item $\langle \mathcal{K}, \Omega, \mathit{z}, 2^{|\mathit{z}|}\rangle$,
	      если наблюдаемыми являются лишь события, одразующие разбиение $\mathcal{z}(\Omega)$.
\end{itemize}
где $\mathcal{K}$ --- комплекс основных условий и действий, характерихующих эксперимент $G$ и позволяющих его многократно повторять.

\subsection{Событие, достоверное и невозможное}
Событие ($A, B, C_1, ... $) --- произвольный набор элементарных исходов,
или, другими словами, произвольное подмножество множества элементарных исходов. \\
Если $A = \Omega$, то $A$ --- достоверное событие. \\
$\varnothing$ --- невозможное событие, не происходит никогда, не содержит ни одного элементарного исхода.

\subsection{Операции над множествами и событиями}
\begin{itemize}
	\item Пересечение \\
	      \[C = A \cap B\]
	      или
	      \[C = A \cdot B\]
	\item Объединение \\
	      \[C = A \cup B\]
	      или
	      \[C = A + B\]
	\item Разность \\
	      \[C = A \setminus B\]
	\item Симметрическая разность \\
	      \[C = A \triangle B = (A \setminus B) \cup (B \setminus A) = (A \cup B) \setminus (A \cap B)\]
	\item Дополнение \\
	      \[\overline{A} = \Omega \setminus A\]
	\item Формулы де Моргана \\
	      \[A \cap B = \overline{\overline{A} \cup \overline{B}}\]
	      \[A \cup B = \overline{\overline{A} \cap \overline{B}}\]
\end{itemize}

\section{Тема 2. Сигма-алгебра событий и вероятностное пространство}

\subsection{Алгебра конечных подмножеств и алгебра событий}
Алгебра событий $\mathcal{A}$ --- непустая схема подмножеств $\Omega$, удовлетворяющая следующим аксиомам:
\begin{itemize}
	\item[A1:] $A \in \mathcal{A} \Rightarrow \overline{A} \in \mathcal{A}$
	\item[A2:] $A,B \in \mathcal{A} \Rightarrow A \cup B \in \mathcal{A}$
\end{itemize}

Свойства:
\begin{enumerate}
	\item $A,B \in \mathcal{A} \Rightarrow A \setminus B \in \mathcal{A}, A \triangle B \in \mathcal{A}$
	\item $\Omega \in \mathcal{A}, \varnothing \in \mathcal{A}$
	\item $A \cap B \in \mathcal{A}$
	\item $\mathcal{A}$ --- замкнутая система подмножеств $\Omega$ относительно конечного числа теоретико-множественных операций
\end{enumerate}

\subsection{Сигма-алгебра счетных подмножеств и сигма-алгебра событий}
Для пространства элементарных исходов, не являющимся конечным, не достаточно ограничиться алгебрами.
Таким образом $\sigma$-алгебра событий $\mathcal{B}$ --- непустой класс подмножеств $\Omega$, удовлетворяющая следующим аксиомам:
\begin{itemize}
	\item[A1:] $A \in \mathcal{B} \Rightarrow \overline{A} \in \mathcal{B}$
	\item[A2$'$:] $A_i \in \mathcal{B}, i = 1,2,... \Rightarrow \bigcup^{\infty}_{i = 1}A_i \in \mathcal{B}$
\end{itemize}

Свойства:
\begin{enumerate}
	\item $\Omega \in \mathcal{A}, \varnothing \in \mathcal{A}$
	\item $\bigcap^{\infty}_{i=1}A_i \in \mathcal{B}$
	\item Любая $\sigma$-алгебра событий является одновременно и алгеброй событий
	\item Если пространство элементарных исходов $\Omega$ конечно,
	      то любая алгебра событий является $\sigma$-алгеброй событий.
	\item $\mathcal{B}$ --- замкнутая система подмножеств $\Omega$ относительно конечного числа теоретико-множественных операций
\end{enumerate}

\subsection{Сигма-алгебра борелевских подмножеств и борелевская сигма-алгебра событий}
Пусть $\Omega = (-\infty, \infty)$ \\
Минимальная $\sigma$-алгебра, которой принадлежат всевозможные интервалы $[a, b], [a, b), (a, b], (a, b)$
носит название \textbf{борелевской $\sigma$-алгебры}.

\subsection{Измеримое пространнство, мера множества и ее свойства}
\subsection{Сигма-аддитивная мера множества}
\subsection{Вероятностная мера (вероятность)}
\subsection{Пространство с мерой}
\subsection{Вероятностное пространство}
Вероятностной моделью эксперимента $G$ с конечным множеством $\Omega$
равновозможных исходов служит система объектов
\[\langle \Omega; P(\omega) = \frac{1}{|\Omega|}, \omega \in \Omega\]
называется \textbf{вероятностным пространством}.

\section{Тема 3. Определение вероятности и ее свойства}

\subsection{Классическая вероятность}
Если $\Omega = \{\omega_1, ... \omega_n\}, 1 \leq n < \infty$, и все элементарные события наблюдаемы и равнозначны,
то вероятность эдементарного исхода:
\[P(\omega_i) = \frac{1}{n}, i = 1, ..., n\]
В классической схеме вероятность любого события $A = \{\omega_{i_1}, ..., \omega_{i_n}\} \subset \Omega$ определяется
как отношение числа $m$ благоприятных для события $A$ элементарных исходов
к общему числу элементарных исходов $n$. \\
\[P(A) = \frac{|A|}{|\Omega|} = \frac{m}{n}, 0 \leq m < n\]

\subsection{Свойства класической вероятности}

\subsection{Геометрическая вероятность}
% Todo: expand reasoning here
Пусть $\Omega$ --- некоторая область, имеющая меру $\mu(\Omega)$ (длину, площадь, объем и т.д.), такую,
что $0 < \mu(\Omega) < \infty$.
Скажем,что точка равномерным образом попадает в $\Omega$ (реализуется принцип геометрической вероятности),
если вероятность $P(A)$ попадания ее в каждую область $A$, являющуюся подобластью $\Omega$,
пропорциональна мере этой области $\mu(A)$. \\
\[P(A) = \frac{\mu(A)}{\mu(\Omega)}\]

\subsection{Статистическое определение вероятности}
При достаточно больших повторениях $N_1, N_2, ...$ экспериментов $G_1, G_2, ...$ относительные
частоты события $A$ называются статистически устойчивым, и колеблются вокург некоторого значения $P(A)$.
$P(A)$ называется вероятностью события $A$, а оценкой $P(A)$ служит статистическая вероятность
\[\widehat{P_N}(A) = \frac{\sum^m_{i=1}N_{i(A)}}{\sum^m_{i=1}N_i}\]

\subsection{Пример статистического расчета вероятности}

\subsection{Модель случайного эксперимента в виде вероятностного пространства}

\subsection{Аксиоматическое определение вероятности. Система аксиом А.Н. Колмогорова}
\begin{itemize}
	\item[P1.] $P(A) \geq 0$ --- аксиома неотрицательности;
	\item[P2.] $P(\Omega) = 1$ --- аксиома нормированности;
	\item[P3.] $P(A + B) = P(A) + P(B)$ --- аксиома сложения, если
		$A, B \in \mathcal{B}, A \cup B = \varnothing$;
	\item[P3$'$.] $P(A_1 + ... + A_n + ...) = P(A_1) + ... + P(A_n) + ...$ ---
		расширенная аксиома сложения в случае произвольного (не обязательно конечного)
		пространства элементарных исходов для счетного числа попарно несовместных событий.
\end{itemize}

\subsection{Свойства вероятности}
\begin{enumerate} % Todo: fill in the reasoning
	\item $P(\overline{A}) = 1 - P(A)$
	\item $P(\varnothing) = 0$
	\item $P(A) \leq P(B)$ если $A \subset B$
	\item $0 \leq P(A) \leq 1$
	\item $P(A \cup B) = P(A) + P(B) - P(AB)$
	\item $P(A_1 \cup ... \cup A_n) = P(A_1) + ... + P(A_n) - P(A_1A_2) - P(A_1A_3) - ... - P(A_{n-1}A_n) + P(A_1A_2A_3) + ... + (-1)^{n+1}P(A_1A_2...A_n)$
	\item $P(A_1 + ... + A_n) = P(A_1) + ... + P(A_n)$
\end{enumerate}

\section{Тема 4. Условная вероятность и независимость событий}

\subsection{Условная вероятность}
Условную вероятность $P(B|A)$ события $B$ при условии события $A$ в рамках классической схемы
естественно определить как отношение числа исходов $m_{AB}$,
благоприятных для совместного осуществления событий $A$ и $B$,
к числу исходов $m_A$, благоприятных для события $A$, т. е.
\[P(B|A) = \frac{m_{AB}}{m_A} = \frac{\frac{m_{AB}}{n}}{\frac{m_A}{n}} = \frac{P(AB)}{P(A)}, n = |\Omega|\]
Условная вероятность обладает всеми свой-свойствами безусловной вероятности. Так,
\[P(\Omega|A) = 1, P(\varnothing|A) = 0, P(C+B|A) = P(C|A) + P(B|A)\]
Также
\[P(\overline{B}|A) + 1 - P(B|A)\]
вытекает из
\[1 = P(\Omega|A) = P(B + \overline{B}|A) = P(B|A) + P(\oveline{B}|A)\]

\subsection{Формула умножения вероятностей}
\[P(A_1A_2...A_n) = P(A_1)P(A_2|A_1)P(A_3|A_1A_2)...P(A_n|A_1A_2...A_{n-1})\]

Получение
\begin{gather*}
	P(A_n|A_1A_2...A_{n-1}) = \frac{P(A_1A_2...A_n)}{P(A_1A_2...A_{n-1})} \\
	P(A_1A_2...A_n) = P(A_n|A_1A_2...A_{n-1}) \times P(A_1A_2...A_{n-1})
\end{gather*}
аналогично
\begin{gather*}
	P(A_1A_2...A_{n-1}) = P(A_n|A_1A_2...A_{n-2}) \times P(A_1A_2...A_{n-2})
	P(A_1A_2...A_{n-2}) = P(A_n|A_1A_2...A_{n-3}) \times P(A_1A_2...A_{n-3})
	\vdots
	P(A_1A_2) = P(A_2|A_1) \times P(A_1)
\end{gather*}
Подставляем все в одну формулу.

\subsection{Независимость событий}
События $A$ и $B$ называются независимыми,
если условная вероятность события $B$ при условии $A$
совпадает с безусловной вероятностью события $B$, т.е.
\[P(B|A) = P(B)\]
Независимость событий $A$ и $B$ эквивалентна выполнению равенства
\[P(AB) = P(A)P(B)\]
Назовем события $A$, $B$ и $C$ независимыми (в совокупности), если
$P(AB) = P(A)P(B), P(AC) = P(A)P(C), P(BC) = P(B)P(C), P(ABC) = P(A)P(B)P(C)$

\subsection{Последовательное и параллельное соединение элементов в структурных схемах наадежности}
\begin{itemize}
	\item Последовательное соединеие: \\
	      Операция: $\cup$ \\
	      Формула: $P(A) = 1 - [1 - P(A_1)]...[1 - P(A_n)]$ (это формула де Моргана)
	\item Параллельное соединеие: \\
	      Операция: $\cap$ \\
	      Формула: $P(A) = P(A_1)...P(A_n)$
\end{itemize}

\subsection{<<Обобщенное>> соединение элементов в структурных схемах надежности}
\subsection{Задача приближенного расчета вероятности блокировки в проводной сети (метод просеянной нагрузки)}
\subsection{Формула полной вероятности}
\subsection{Формула Байеса}

\section{Тема 5. Схема Бернулли и предельные теоремы}
\subsection{Схема Бернулли}
\subsection{Формула Бернулли}
\subsection{Локальная предельная теорема Пуассона и формула Пуассона}
\subsection{Локальная предельная теорема Муавра-Лапласа и локальная формула Муавра-Лапласа}
\subsection{Интегральная предельная теорема Муавра-Лапласа и интегральная формула Муавра-Лапласа}
\subsection{Применение приближенных формул Пуассона и Муавра-Лапласа}
\subsection{Теорема Бернулли и слабый закон простых чисел}
\subsection{Полиномиальная схема и полиномиальное распределение}

\end{document}

\documentclass[12pt]{article}
\usepackage[T1, T2A]{fontenc}
\usepackage[utf8]{inputenc}
\usepackage[russian]{babel}
\usepackage{hyperref}
\usepackage{datetime}
\usepackage{amsmath}
\usepackage{amsfonts}
\usepackage{tikz}
\graphicspath{ {./Images/} }

% For slope Fields
\usepackage{pgfplots}
\usetikzlibrary{calc}
\usetikzlibrary{shapes.geometric, arrows.meta}
\pgfplotsset{compat=1.8}

\author{Григорий Матюхин}
\date{\today}
\title{
	Теория вероятностей и математическая статистика \\
	\large Подготовка к контрольной работе \textnumero1.
}

\begin{document}
\maketitle
\newpage
\tableofcontents
\newpage

\subsection{Задание \textnumero1.}
Вероятность правильной передачи символа по каналу связи равна $p = 0.9$, причем известно, что каждый символ искажается независимо от остальных.
Случайная величина $\xi$ --- число правильно переданных символов в сообщении из $n = 7$ символов. Найдите:
\begin{enumerate}
	\item Ряд распределения случайной величины $\xi$;
	\item Функцию распределения случайной величины $\xi$ и постройте ее график;
	\item Вероятность попадания случайной величины $\xi$ в интервал $(x_1, x_2],$ $ x_1 = 2, x_2 = 5$;
	\item Найдите ряд распределения случайных величин $\eta = a(\xi - b)^2 + c,$ $ a = -4, b = 7, c = -2$.
\end{enumerate}
\subsubsection*{Решение}
\begin{enumerate}
	\item Найдем ряд распределения случайной величины $\xi$: \\
	      $\xi(m) = C_n^mp^m(1-p)^{n-m}, m = \overline{0, n}$ --- закон распределения случайной величины. \\
	      Тогда ряд распределения
	      \begin{center}
		      \begin{tabular}{ |c|c|c|c|c|c|c|c|c| }
			      \hline
			      $\xi$ & 0         & 1                    & 2                    & 3                    & 4                    & 5    & 6    & 7    \\
			      \hline
			      $P$   & $10^{-7}$ & $6.3 \times 10^{-6}$ & $1.7 \times 10^{-4}$ & $2.6 \times 10^{-3}$ & $2.3 \times 10^{-2}$ & 0.12 & 0.37 & 0.48 \\
			      \hline
		      \end{tabular}
	      \end{center}
	\item Функция распределения: \\
	      $F_{\xi}(x) = P\{\xi < x\}$ \\
	      \begin{tikzpicture}
		      \begin{axis}[
				      axis lines = middle,
				      xlabel = {$x$},
				      ylabel = {$F(x)$},
				      ymode = log,
				      xmin=0, xmax=8,
				      ymin = 1/10^8, ymax=10]

			      \addplot [Latex-, ultra thick, blue, domain = 0:1] {0.0000001};
			      \addplot [Latex-, ultra thick, blue, domain = 1:2] {0.0000063};
			      \addplot [Latex-, ultra thick, blue, domain = 2:3] {0.0000063 + 0.0001701};
			      \addplot [Latex-, ultra thick, blue, domain = 3:4] {0.0000063 + 0.0001701 + 0.0025515};
			      \addplot [Latex-, ultra thick, blue, domain = 4:5] {0.0000063 + 0.0001701 + 0.0025515 + 0.0229635};
			      \addplot [Latex-, ultra thick, blue, domain = 5:6] {0.0000063 + 0.0001701 + 0.0025515 + 0.0229635 + 0.124003};
			      \addplot [Latex-, ultra thick, blue, domain = 6:7] {0.0000063 + 0.0001701 + 0.0025515 + 0.0229635 + 0.124003 + 0.372009};
			      \addplot [Latex-, ultra thick, blue, domain = 7:8] {1};
		      \end{axis}
	      \end{tikzpicture}
	\item Вероятность попадания случайной величины $\xi$ в интервал $(2; 5]$:
	      \begin{gather*}
		      P\{2 < \xi \leq 5\} = F(2) + F(3) + F(4) = 1.7 \times 10^{-4} + 2.6 \times 10^{-3} + 2.3 \times 10^{-2} = 0.02577
	      \end{gather*}
	\item Ряд распределения $\eta = -4(\xi - 7)^2 - 2$: \\
	      \begin{center}
		      \begin{tabular}{ |c|c|c|c|c|c|c|c|c| }
			      \hline
			      $\eta$ & -198      & -146                 & -102                 & -66                  & -38                  & -18  & -6   & -2   \\
			      \hline
			      $P$    & $10^{-7}$ & $6.3 \times 10^{-6}$ & $1.7 \times 10^{-4}$ & $2.6 \times 10^{-3}$ & $2.3 \times 10^{-2}$ & 0.12 & 0.37 & 0.48 \\
			      \hline
		      \end{tabular}
	      \end{center}
\end{enumerate}

\subsection{Задание \textnumero2.}
Непрерывная случайная величина $\xi$ задана плотностью распределения $p(x)$:
\[
	p(x) =
	\begin{cases}
		Ax^2, -3 < x \leq 3 \\
		0, x \leq -3, x > 3
	\end{cases}
\]
Найдите:
\begin{enumerate}
	\item Константу $A$;
	\item Функцию распределения случайной величины $F(x)$ и постройте ее график;
	\item Вычислите плотность распределения случайной величины $\eta = a\xi^3 + c,$ $a = 2, c = -4$;
	\item Вычислите плотность распределения случайной величины $\mu = a(\xi - b)^2 + c,$ $a = 2, b = 0, c = -4$.
\end{enumerate}
\subsubsection*{Решение}
\begin{enumerate}
	\item Поиск константы: \\
	      \begin{gather*}
		      \int_{-\infty}^{\infty}p(x)dx = 1 \\
		      \int_{-\infty}^{-3}0dx + \int_{-3}^{3}Ax^2dx + \int_{3}^{\infty}0dx = 1 \\
		      \int_{-3}^{3}Ax^2dx = 1 \\
		      A\int_{-3}^{3}x^2dx = 1 \\
		      A\left.\frac{x^3}{3}\right|^3_{-3} = 1 \\
		      A\left(\frac{3^3}{3} - \frac{(-3)^3}{3}\right) = 1 \\
		      A\left(9 - (-9)\right) = 1 \\
		      18A = 1 \\
		      A = \frac{1}{18} \\
	      \end{gather*}
	      Тогда плотность распределения $p(x)$
	      \[
		      p(x) =
		      \begin{cases}
			      \frac{1}{18}x^2, -3 < x \leq 3 \\
			      0, x \leq -3, x > 3
		      \end{cases}
	      \]
	\item Поиск функции распределения: \\
	      \begin{enumerate}
		      \item $x \leq -3$:
		            \begin{gather*}
			            F(x) = \int_{-\infty}^{-3}0du = 0
		            \end{gather*}
		      \item $-3 < x \leq 3$:
		            \begin{gather*}
			            F(x) = \int_{-3}^3 \frac{1}{18}u^2du = \\
			            = \frac{1}{18}\int_{-3}^3u^2du = \\
			            = \frac{1}{18}\left(\left.\frac{u^3}{3}\right|_{-3}^{3}\right) = \\
			            = \frac{1}{18}\left(\frac{3^3}{3} - \frac{(-3)^3}{3}\right) = \\
			            = \frac{1}{18}\left(9 - (-9)\right) = \\
			            = \frac{1}{18} \times 18 = 1
		            \end{gather*}
		      \item $x > 3$:
		            \begin{gather*}
			            F(x) = \int_{3}^x0du = 0
		            \end{gather*}
	      \end{enumerate}
	      \begin{center}
		      \begin{tikzpicture}
			      \begin{axis}[
					      axis lines = middle,
					      xlabel = {$x$},
					      ylabel = {$F(x)$},
					      xmin=-5, xmax=5,
					      ymin=-4, ymax=4]

				      \addplot[domain = -5:-3, smooth, blue, ultra thick] {0};
				      \addplot[domain = 3:5, smooth, blue, ultra thick] {0};
				      \addplot[domain = -3:3, smooth, blue, ultra thick] {1};
			      \end{axis}
		      \end{tikzpicture}
	      \end{center}
	\item Плотность распределения случайной величины $\eta = 2\xi^3 - 4$: \\
	      Пусть $P_{\eta}(y)$ --- плотность распределения случайной величины $\eta$.
	      Так как $\eta = g(x) = 2x^3 - 4$ --- монотонна \\
	      \[x = \sqrt[3]{\frac{y}{2} + 2} = g^{-1}(y)\]
	      Тогда
	      \begin{gather*}
		      P_{\eta}(y) = P(g^{-1}(y))\times \left|(g^{-1}(y))'\right| \\
		      P_{\eta}(y) = P\left(\sqrt[3]{\frac{y}{2} + 2}\right)\times \left|(\sqrt[3]{\frac{y}{2} + 2})'\right| \\
		      P_{\eta}(y) = P\left(\sqrt[3]{\frac{y}{2} + 2}\right)\times \left|\frac{1}{6}\left(\frac{y}{2} + 2 \right)^{-\frac{2}{3}}\right| \\
	      \end{gather*}
	\item Плотность распределения случайной величины $\mu = 2\xi^2 - 4$:
\end{enumerate}

\end{document}

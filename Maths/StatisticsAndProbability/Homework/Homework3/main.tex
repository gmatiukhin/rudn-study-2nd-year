\documentclass[12pt]{article}
\usepackage[T1, T2A]{fontenc}
\usepackage[utf8]{inputenc}
\usepackage[russian]{babel}
\usepackage{hyperref}
\usepackage{datetime}
\usepackage{amsmath}
\usepackage{amsfonts}
\usepackage{tikz}
\graphicspath{ {./Images/} }

% For slope Fields
\usepackage{pgfplots}
\usetikzlibrary{calc}
\usetikzlibrary{shapes.geometric, arrows.meta, positioning, intersections, fillbetween}
\pgfplotsset{compat=1.8}
\usepgfplotslibrary{fillbetween}

\usepackage[14pt]{extsizes} % для того чтобы задать нестандартный 14-ый размер шрифта
\usepackage[utf8]{inputenc}
\usepackage[russian]{babel} % поддержка русского языка
\usepackage{amsmath}  %  математические символы
\usepackage[left=20mm, top=15mm, right=15mm, bottom=30mm, footskip=15mm]{geometry} % настройки полей документа
\usepackage{indentfirst} % по умалчанию убирается отступ у первого абзаца в секции, это отменяет это.
\usepackage{paralist} % добавить компактные списки (compactitem, compactenum, compactdesc)

\usepackage{fancyvrb}
\usepackage{framed}

\DeclareMathOperator{\cov}{cov}

\begin{document}
% \maketitle
\begin{titlepage}

	\begin{center}
		\hfill \break
		\textbf{
			\large{РОССИЙСКИЙ УНИВЕРСИТЕТ ДРУЖБЫ НАРОДОВ}\\
			\normalsize{Факультет физико-математических и естественных наук}\\
			\normalsize{Кафедра прикладной информатики и теории вероятностей}\\
		}
		\vspace*{\fill}
		\Large{\textbf{Индивидуальное домашнее задание \textnumero 3}}
		\\
		\underline{\textit{\normalsize{Дисциплина: Теория вероятностей и математическая статистика}}}
		\vspace*{\fill}

	\end{center}

	\begin{flushright}
		Студент: \underline{Григорий Матюхин}\\ \vspace{0.5cm}
		Группа: \underline{НПИбд-01-21}
	\end{flushright}


	\begin{center} \textbf{МОСКВА} \\ 2023 г. \end{center}
	\thispagestyle{empty} % выключаем отображение номера для этой страницы

\end{titlepage}
\newpage
\tableofcontents
\newpage

\section{Варианты задач}

\begin{tabular}{|c|c|c|c|c|c|c|}
	\hline
	1 & 2  & 3 & 4  & 5 & 6  & 7 \\
	\hline
	1 & 32 & 6 & 19 & 6 & 24 & 6 \\
	\hline
\end{tabular}

\section*{Задание \textnumero1.}
\addcontentsline{toc}{section}{Задание \textnumero1.}

Дискретная двумерная случайная величина $(\xi, \eta)$ задана рядом распределения.
Найдите:

\begin{enumerate}
	\item ряды распределения случайных величин $\xi$ и $\eta$;
	\item вероятности $P\{-1 \leq \xi \leq 7, 0 \leq \eta \leq 15\}$;
	\item условное распределение случайной величины $\eta$ при условии $\xi = 1$;
	\item ряд распределения случайной величины $\mu = \eta - 2 \sqrt{\xi} + 1$.
\end{enumerate}

\begin{tabular}{|c|c|c|c|}
	\hline
	$\xi \backslash \eta$ & 10   & 14   & 18   \\
	\hline
	1                     & 0.25 & 0.15 & 0.32 \\
	\hline
	9                     & 0.1  & 0.05 & 0.13 \\
	\hline
\end{tabular}

\subsubsection*{Решение}
\begin{enumerate}
	\item \mbox{}\\
	      % sum of probabilities in rows
	      \begin{tabular}{|c|c|c|}
		      \hline
		      $\xi $ & $1$    & $9$    \\
		      \hline
		      $p$    & $0.72$ & $0.28$ \\
		      \hline
	      \end{tabular}
	      % sum of probabilities in columns
	      \begin{tabular}{|c|c|c|c|}
		      \hline
		      $\eta $ & $10$   & $14$  & $18$   \\
		      \hline
		      $p$     & $0.35$ & $0.2$ & $0.45$ \\
		      \hline
	      \end{tabular}

	\item \mbox{}\\
	      \begin{gather*}
		      P\{-1 \leq \xi \leq 7, 0 \leq \eta \leq 15\} = 0.25 + 0.15 = 0.4
	      \end{gather*}

	\item \mbox{}\\
	      \begin{tabular}{|c|c|c|c|}
		      \hline
		      $\xi \backslash \eta$ & 10   & 14   & 18   \\
		      \hline
		      % 1                     & 0.25/0.72 & 0.15/0.72 & 0.32/0.72 \\
		      1                     & 0.35 & 0.21 & 0.44 \\
		      \hline
		      % % 9                     & 0.1/0.28  & 0.05/0.28 & 0.13/0.28 \\
		      % 9                     & 0.36 & 0.18 & 0.46 \\
		      % \hline
	      \end{tabular}

	\item \mbox{}\\
	      \[\mu = g(\xi, \eta) = \eta - 2 \sqrt{\xi} + 1\]
	      \begin{gather*}
		      g(1, 10) = 9 \\
		      g(1, 14) = 13 \\
		      g(1, 18) = 18 \\
		      g(9, 10) = 5 \\
		      g(9, 14) = 9 \\
		      g(9, 18) = 1
	      \end{gather*}
	      \begin{tabular}{|c|c|c|c|c|c|c|}
		      \hline
		      $\mu$ & 9    & 3    & 18   & 5   & 9    & 13   \\
		      \hline
		            & 0.25 & 0.15 & 0.32 & 0.1 & 0.05 & 0.13 \\
		      \hline
	      \end{tabular} \\
	      \begin{tabular}{|c|c|c|c|c|}
		      \hline
		      $\mu$ & 5   & 9   & 13   & 18   \\
		      \hline
		      $p$   & 0.1 & 0.3 & 0.28 & 0.32 \\
		      \hline
	      \end{tabular}
\end{enumerate}

\section*{Задание \textnumero 2.}
\addcontentsline{toc}{section}{Задание \textnumero 2.}

Задана плотность совместного распределения непрерывной двумерной случайной величины $(\xi,\eta)$:

\begin{equation*}
	p_{\xi,\eta}(x,y) =
	\begin{cases}
		\begin{aligned}[t]
			C(x + 2y), & (x;y) \in D    \\
			0,         & (x;y) \notin D
		\end{aligned}
	\end{cases}
\end{equation*}

где область $D$ ограниченна графиками функций $y = \sqrt{x}$, $x = 9$ и осью абсцисс.
Найдите (в пунктах 4,5,6 расставить пределы интегрирования, интеграл не вычислять):

\begin{enumerate}
	\item значение постоянной $C$;
	\item частную плотность распределения случайной величины $\xi$;
	\item условную плотность распределения случайной величины $\eta$ при условии $\xi$;
	\item значение совместной функции распределения $F_{\xi\eta}(x,y)$ в точке $(2; 4)$;
	\item вероятность попадания с.в. $(\xi, \eta)$ в область: $3 - x \leq y \leq \frac{1}{x}$;
	\item значение функции распределения $F_{\mu}(z)$ случайной величины $\mu = -\eta + (\xi - 1)^2$ в точке $z = -4$.
\end{enumerate}

\begin{tikzpicture}
	\begin{axis}[
			axis lines = middle,
			xlabel = {$x$},
			xtick distance = 1,
			ylabel = {$y$},
			xmin=0, xmax=10,
			ymin=0, ymax=5]

		\addplot[name path=SQ,domain=0:9,smooth, blue, ultra thick] {x^(1/2)};
		\addplot[draw=none,name path=H, domain=0:9] {0};     % “fictional” curve
		\addplot[gray] fill between[of=SQ and H,soft clip={domain=0:9}]; % filling

		\addplot[name path=A,domain=2.6:9,smooth, red, ultra thick] {1/x};
		\addplot[draw=none,name path=B, domain=0:9, red] {3-x};     % “fictional” curve
		\addplot[red] fill between[of=A and B,soft clip={domain=2.62:9}]; % filling

		\path[name intersections={of=A and B,by=M}];
		\draw (M) node[above=.5em] {$(2.62, 0.38)$} circle[radius=2pt];

	\end{axis}
\end{tikzpicture}

\subsubsection*{Решение}

\begin{enumerate}
	\item \mbox{}\\
	      Из условия нормировки
	      \begin{gather*}
		      %https://www.wolframalpha.com/input?i=Integrate%5BIntegrate%5B%28x+%2B+2y%29%2C%7By%2C0%2CSqrt%5Bx%5D%7D%5D%2C%7Bx%2C0%2C3%7D%5D
		      \int_{-\infty}^{\infty}\int_{-\infty}^{\infty}p(x,y)dxdy = 1 \Rightarrow \iint_{D}p(x,y)dxdy = 1 \\
		      \iint_{D}C(x + 2y)dxdy = \int_0^9\int_0^{\sqrt{x}}C(x + 2y)dydx = \\
		      = C\int_0^9\int_0^{\sqrt{x}}(x + 2y)dydx = \\
		      \cdots \\
		      = C \times 10.735 \\
		      C \times 10.735 = 1 \Rightarrow C = 0.09 \Rightarrow \\
		      p_{\xi,\eta}(x,y) =
		      \begin{cases}
			      \begin{aligned}[t]
				      0.09(x + 2y), & (x;y) \in D    \\
				      0,            & (x;y) \notin D
			      \end{aligned}
		      \end{cases},
	      \end{gather*}

	\item \mbox{}\\
	      \begin{gather*}
		      %https://www.wolframalpha.com/input?i=%5Cint_%7B0%7D%5E%7B%5Csqrt%7Bx%7D%7D0.09+*+%28x+%2B+2y%29dy
		      p_{\xi}(x) = \int_{-\infty}^{\infty}p_{\xi,\eta}(x,y)dy = \\
		      = \int_{0}^{\sqrt{x}}0.09 * (x + 2y)dy = 0.09(x^{3/2} + x) \Rightarrow \\
		      p_{\xi}(x) =
		      \begin{cases}
			      \begin{aligned}[t]
				      0.09(x^{3/2} + x) & x \in D    \\
				      0                 & x \notin D
			      \end{aligned}
		      \end{cases}
	      \end{gather*}

	\item \mbox{}\\
	      \begin{gather*}
		      p_{\eta}(y|\xi = x) = \frac{\partial}{\partial y}F_{\eta}(y|\eta = x)
		      = \frac{p_{\xi, \eta}(x, y)}{p_{\xi}(x)} \Rightarrow \\
		      p_{\eta}(y|\xi = x) =
		      \begin{cases}
			      \begin{aligned}[t]
				      \frac{(x + 2y)}{(x^{3\div2} + x)}, & (x;y) \in D \\
				      0,                                 & (x;y) \notin D
			      \end{aligned}
		      \end{cases}
	      \end{gather*}

	\item \mbox{}\\
	      \begin{gather*}
		      F(2,10) = \int_{-\infty}^{10}dy\int_{-\infty}^{2}p(x, y)dx = \\
		      = \int_{0}^{\sqrt{2}}dy\int_{y^2}^{2}p(x, y)dx
	      \end{gather*}

	\item \mbox{}\\
	      \begin{gather*}
		      P\{a_1 < \xi < b_1, a_2 < \eta < b_2\} = \int_{a_1}^{b_1}dx\int_{a_2}^{b_2}p(x, y)dy = \\
		      % NOTE: (2.62; 0.38) --- the corner
		      \int_{2.62}^{3}dx\int_{3 - x}^{1/{x}}p(x, y)dy +
		      \int_{3}^{9}dx\int_{0}^{1/{x}}p(x, y)dy
	      \end{gather*}

	\item \mbox{}\\
	      \begin{gather*}
		      \mu = -\eta + (\xi - 1)^2 \Rightarrow g(x, y) = -y + (x - 1)^2 \\
		      F_{\mu}(z) = \iint_{g(x, y) < z} p(x,y)dxdy
		      p_{\mu}()
		      % TODO: z = -4
	      \end{gather*}

\end{enumerate}

\section*{Задание \textnumero 3.}
\addcontentsline{toc}{section}{Задание \textnumero 3.}

Дискретная двумерная случайная величина $(\xi, \eta)$ задана рядом распределения.
Найдите:

\begin{enumerate}
	\item ряд распределения случайных величин $\xi$ и $\eta$;
	\item математическое ожидание и дисперсию случайных величин $\xi$ и $\eta$;
	\item ковариацию и коэффициент корреляции случайных величин $\xi$ и $\eta$;
	\item математическое ожидание и дисперсию случайной величины $\mu = 2(\xi - 2\eta) + 3(\eta - 2\xi)$;
	\item ковариацию случайных величин $\xi$ и $\mu$.
\end{enumerate}

\begin{tabular}{|c|c|c|c|}
	\hline
	$\xi \backslash \eta$ & 2    & 3    & 4    \\
	\hline
	-2                    & 0.05 & 0.36 & 0.01 \\
	\hline
	4                     & 0.09 & 0.31 & 0.18 \\
	\hline
\end{tabular}

\subsubsection*{Решение}

\begin{enumerate}
	\item \mbox{}\\
	      % sum of probabilities in rows
	      \begin{tabular}{|c|c|c|}
		      \hline
		      $\xi $ & -2   & 4    \\
		      \hline
		      $p$    & 0.42 & 0.58 \\
		      \hline
	      \end{tabular}
	      % sum of probabilities in columns
	      \begin{tabular}{|c|c|c|c|}
		      \hline
		      $\eta $ & 2    & 3    & 4    \\
		      \hline
		      $p$     & 0.14 & 0.67 & 0.19 \\
		      \hline
	      \end{tabular}

	\item \mbox{}\\
	      \begin{gather*}
		      M\xi = \sum_iX_ip_i = -2 \times 0.42 + 4 \times 0.58 = 1.48 \\
		      M\eta = \sum_jX_jp_j = 2 \times 0.14 + 3 \times 0.67 + 4 \times 0.19 = 3.05
	      \end{gather*}

	\item \mbox{}\\
	      \begin{gather*}
		      \cov(\xi, \eta) = M\mathring{\xi}\mathring{\eta} = M[(\xi - M\xi)(\eta - M\eta)] = \\
		      = \sum_{i,j}(X_i - M\xi)(Y_j - M\eta)p_{ij} = 0.366
		      \\\\
		      D\xi = M\xi^2 - (M\xi)^2 = \sum_iX_i^2p_i - (M\xi)^2 = 9.48 \\
		      D\eta = M\eta^2 - (M\eta)^2 = \sum_jX_j^2p_j - (M\eta)^2 = 6.57
		      \\\\
		      \rho = \frac{\cov(\xi, \eta)}{\sqrt{D\xi \times D\eta}} = \\
		      \frac{0.366}{\sqrt{9.48 \times 6.57}} = 0.17
	      \end{gather*}

	\item \mbox{}\\
	      \[\mu = g(\xi, \eta) = 2(\xi - 2\eta) + 3(\eta - 2\xi)\]
	      \begin{gather*}
		      g(-2, 2) = 6 \\
		      g(-2, 3) = 5 \\
		      g(-2, 4) = 4 \\
		      g(4, 2) = -18 \\
		      g(4, 3) = -19 \\
		      g(4, 4) = -20
	      \end{gather*}
	      \begin{tabular}{|c|c|c|c|c|c|c|}
		      \hline
		      % TODO: place g comps on different lines, leave a note about precision loss
		      $\mu$ & 6    & 5    & 4    & -18  & -19  & -20  \\
		      % $\mu$ & 6    & 5    & 4    & -18  & -19  & -20  \\
		      \hline
		      $p$   & 0.05 & 0.36 & 0.01 & 0.09 & 0.31 & 0.18 \\
		      \hline
	      \end{tabular} \\

	      \begin{gather*}
		      M\mu = \sum_iX_ip_i = -8.97 \\
		      % WARN: check again
		      D\mu = M\mu^2 - (M\mu)^2 = \sum_iX_i^2p_i - (M\mu)2 = 233.0
	      \end{gather*}

	\item \mbox{}\\

	      \begin{tabular}{|c|c|c|c|c|c|c|}
		      \hline
		      $\xi \backslash \mu$ & 6    & 5    & 4     & -18  & -19  & -20  \\
		      \hline
		      % -2 & 0.42 * 0.05 & 0.42 * 0.36 & 0.42 * 0.01 & 0.42 * 0.09 & 0.42 * 0.31 & 0.42 * 0.18 \\
		      -2                   & 0.02 & 0.15 & 0.004 & 0.04 & 0.13 & 0.07 \\
		      \hline
		      % 4 & 0.58 * 0.05 & 0.58 * 0.36 & 0.58 * 0.01 & 0.58 * 0.09 & 0.58 * 0.31 & 0.58 * 0.18 \\
		      4                    & 0.02 & 0.2  & 0.006 & 0.05 & 0.18 & 0.1  \\
		      \hline
	      \end{tabular} \\

	      \begin{gather*}
		      \cov(\xi, \mu) = M\mathring{\xi}\mathring{\mu} = M[(\xi - M\xi)(\mu - M\mu)] = \\
		      = \sum_{i,j}(X_i-M\xi)(Y_j - M\mu)p_{ij} = -0.5
	      \end{gather*}

\end{enumerate}

\section*{Задание \textnumero 4.}
\addcontentsline{toc}{section}{Задание \textnumero 4.}

Задана плотность совместного распределения непрерывной двумерной случайной величины $(\xi, \eta)$:

\begin{equation*}
	p_{\xi\eta} =
	\begin{cases}
		\begin{aligned}[t]
			Ax, & (x;y) \in D    \\
			0,  & (x;y) \notin D
		\end{aligned}
	\end{cases},
\end{equation*}

где область $D\!: y \geq 0, x + y \leq 1,  2y - x \leq 2$.
Найдите:

\begin{enumerate}
	\item значение константы $A$;
	\item математические ожидания случайных величин $\xi$ и $\eta$;
	\item ковариацию случайных величин $\xi$ и $\eta$ (записать интеграл и расставить пределы интегрирования);
	\item математическое ожидание случайной величины $\mu = \max(\eta, -\xi)$ (записать интеграл и расставить пределы интегрирования).
\end{enumerate}

\begin{tikzpicture}
	\begin{axis}[
			axis lines = middle,
			xlabel = {$x$},
			xtick distance = 1,
			ylabel = {$y$},
			xmin=-3, xmax=3,
			ymin=-1, ymax=3]

		\addplot[name path=A,domain=0:1,smooth, blue, ultra thick] {1 - x};
		\addplot[name path=B,domain=-2:0,smooth, blue, ultra thick] {1 + x/2};
		\addplot[name path=H, domain=-2:1,smooth, blue, ultra thick] {0};     % “fictional” curve
		\addplot[gray] fill between[of=A and H,soft clip={domain=-3:3}]; % filling
		\addplot[gray] fill between[of=B and H,soft clip={domain=-3:3}]; % filling

		% \addplot[name path=A,domain=2.6:9,smooth, red, ultra thick] {1/x};
		% \addplot[draw=none,name path=B, domain=0:9, red] {3-x};     % “fictional” curve
		% \addplot[red] fill between[of=A and B,soft clip={domain=2.62:9}]; % filling

		% \path[name intersections={of=A and B,by=M}];
		% \draw (M) node[above=.5em] {$(2.62, 0.38)$} circle[radius=2pt];

	\end{axis}
\end{tikzpicture}


\subsubsection*{Решение}

\begin{enumerate}
	\item \mbox{}\\
	      Из условия нормировки
	      \begin{gather*}
		      %https://www.wolframalpha.com/input?i=%5Cint_%7B-2%7D%5E%7B0%7D%5Cint_%7B0%7D%5E%7B1%2F2%282+%2B+x%29%7D+%28x%29dydx+%2B+%5Cint_%7B0%7D%5E%7B1%7D%5Cint_%7B0%7D%5E%7B1+-+x%7D+%28x%29dydx%
		      \int_{-\infty}^{\infty}\int_{-\infty}^{\infty}p(x,y)dxdy = 1 \Rightarrow \iint_{D}p(x,y)dxdy = 1 \\
		      \iint_{D}Axdxdy =
		      A\left(\int_{-2}^{0}dx\int_{0}^{1/2(2 + x)}xdy + \int_{0}^{1}dx\int_{0}^{1 - x}xdy \right) \\
		      = -0.5 \times A \\
		      -0.5 \times A = 1 \Rightarrow A = -2
	      \end{gather*}

	      или
	      \begin{gather*}
		      %https://www.wolframalpha.com/input?i=%5Cint_%7B0%7D%5E%7B1%7D%5Cint_%7B2y-2%7D%5E%7B1-y%7D+x+dxdy
		      \int_{0}^{1}\int_{2y-2}^{1-y} Ax dxdy = -0.5A \Rightarrow A = -2
	      \end{gather*}

	      \begin{gather*}
		      p_{\xi,\eta}(x,y) =
		      \begin{cases}
			      \begin{aligned}[t]
				      -2x & (x;y) \in D    \\
				      0,  & (x;y) \notin D
			      \end{aligned}
		      \end{cases}
	      \end{gather*}

	\item \mbox{}\\
	      \begin{gather*}
		      p_{\xi}(x) = \int_{-\infty}^{\infty}p_{\xi,\eta}(x,y)dy = \\
		      = \int_{0}^{1/2(2+x)}-2xdy + \int_{0}^{1-x}-2xdy = x^2 - 4x
		      p_{\xi}(x) =
		      \begin{cases}
			      \begin{aligned}[t]
				      x^2-4x, & x \in D    \\
				      0,      & x \notin D
			      \end{aligned}
		      \end{cases} \\\\
		      M\xi = \int_{-\infty}^{\infty}xp_{\xi}(x)dx = \int_{-2}^{1}x(x^2 - 4x)dx = -15.75
	      \end{gather*}
	      \begin{gather*}
		      %https://www.wolframalpha.com/input?i=%5Cint_%7B2y+-+2%7D%5E%7B1+-+y%7D%28-2x%29dx
		      p_{\eta}(y) = \int_{-\infty}^{\infty}p_{\xi,\eta}(x,y)dx = \\
		      = \int_{2y - 2}^{1 - y}-2xdx = 3(y - 1)^2
		      p_{\eta}(y) =
		      \begin{cases}
			      \begin{aligned}[t]
				      3(y - 1)^2, & y \in D    \\
				      0,          & y \notin D
			      \end{aligned}
		      \end{cases} \\\\
		      M\eta = \int_{-\infty}^{\infty}yp_{\eta}(y)dy = \int_{0}^{1}3y(y - 1)^2dy = 0.25
	      \end{gather*}

	\item \mbox{}\\
	      \begin{gather*}
		      \cov(\xi, \eta) = \int_{-\infty}^{\infty}\int_{-\infty}^{\infty}(x - M\xi)(y - M\eta)p_{\xi,\eta}(x,y)dxdy = \\
		      = \int_{0}^{1}\int_{2y-2}^{1-y}(x - M\xi)(y - M\eta)p_{\xi,\eta}(x,y)dxdy
	      \end{gather*}

	\item \mbox{}\\
	      \begin{gather*}
		      \mu = \max(\eta, -\xi) \Rightarrow g(x, y) = \max(x, -y) \\
		      F_{\mu}(z) = \iint_{g(x, y) < z} p(x,y)dxdy \\
		      M\mu = \int_{-\infty}^{\infty}\int_{-\infty}^{\infty}g(x,y)p(x,y)dxdy = \\
		      = \iint_{(x,y) \in D}\max(\eta,-\xi)p(x,y)dxdy \\ % TODO: expand
	      \end{gather*}

\end{enumerate}

\section*{Задание \textnumero 5.}
\addcontentsline{toc}{section}{Задание \textnumero 5.}

Найдите характеристическую функцию непрерывной случайной величины, имеющей плотность распределения:
\begin{equation*}
	p_{\xi}(x) =
	\begin{cases}
		\begin{aligned}[t]
			0,     & x \notin [-1; 1] \\
			x + 1, & x \in [-1; 0]    \\
			1 - x, & x \in [0; 1]
		\end{aligned}
	\end{cases}
\end{equation*}
\subsubsection*{Решение}
\begin{gather*}
	%https://www.wolframalpha.com/input?i=%5Cint_%7B-1%7D%5E%7B0%7De%5E%7Bitx%7D%28x+%2B+1%29dx+%2B+%5Cint_%7B0%7D%5E%7B1%7De%5E%7Bitx%7D%281+-+x%29dx
	f(t) = Me^{it\xi} = \int_{-\infty}^{infty}e^{itx}p_{\xi}(x)dx = \\
	= \int_{-1}^{0}e^{itx}(x + 1)dx + \int_{0}^{1}e^{itx}(1 - x)dx = \\
	= \frac{it - e^{it} + 1}{t^2} + \frac{it + e^{-it} - 1}{t^2} = \frac{e^{it} - e^{-it} + 1}{t^2} = \\
	= \frac{2}{t^2}\left(1 - cos(t)\right)
\end{gather*}

\section*{Задание \textnumero 6.}
\addcontentsline{toc}{section}{Задание \textnumero 6.}

Дисперсия каждой из случайных величин $\xi_i$
(продолжительность работы электролампочки) не превышает 20 часов.
Сколько нужно взять для испытания электролампочек, чтобы вероятность того,
что абсолютное отклонение средней продолжительности горения лампочки
от среднего арифметического их математических ожиданий не превышает 1 часа,
была не меньше $0.95$?
\subsubsection*{Решение}
\begin{gather*}
	D\xi \leq 20 \\ % lets assume it is eq
	\varepsilon = 1 \\
	m = M\left(\frac{1}{n}\sum_{i=1}^{n}\xi_i\right) \\
	% WARN: check comparion signs
	P \left\{\bigg\lvert\frac{1}{n} \sum_{i=1}^{n} \xi_i-m\bigg\rvert \geq \varepsilon\right\} \leq \frac{D\xi}{n\varepsilon^2} \\
	P \left\{\bigg\lvert\frac{1}{n} \sum_{i=1}^{n} \xi_i-m\bigg\rvert < \varepsilon\right\} \geq 1 - \frac{D\xi}{n\varepsilon^2} \\
	P \left\{\bigg\lvert\frac{1}{n} \sum_{i=1}^{n} \xi_i-m\bigg\rvert < 1\right\} \geq 1 - \frac{20}{n} \\
\end{gather*}
\begin{gather*}
	\textup{По условию:} \\
	P \left\{\bigg\lvert\frac{1}{n} \sum_{i=1}^{n} \xi_i-m\bigg\rvert < 1\right\} \geq 0.95 \\
	\textup{Тогда} \\
	1 - \frac{20}{n} \geq 0.95 \Rightarrow n \geq 400
\end{gather*}

\section*{Задание \textnumero 7.}
\addcontentsline{toc}{section}{Задание \textnumero 7.}

Всхожесть семян некоторой культуры равна $0.85$. Оцените при помощи ЦПТ вероятность того,
что из 400 посеянных семян число взошедших будет заключено в пределах от 300 до 380.

\subsubsection*{Решение}
\begin{gather*}
	\textup{По схеме Бернулли:} \\
	p = 0.85, q = 0.15 \\
	\textup{$\xi_i$ --- число успехов в $i$-м испытании} \\
	m = M\xi_i = p = 0.85, \sigma^2 = D\xi_i = pq = 0.1275 \\
	S_n = \xi_1 + \cdots + \xi_n, n = 400 \\
	\varepsilon = (380 - 300) / 2 = 40 \\
	P\left\{\frac{S_n-nm}{\sqrt{n\sigma^2}<x}\right\} \rightarrow[n \to \infty] \Phi(x) \\
	P\left\{ |S_n - np| < \varepsilon \right\} =
	P\left\{-\varepsilon < S_n - np < \varepsilon\right\} = \\
	P\left\{-\frac{\varepsilon}{\sqrt{n\sigma^2}} <
	\frac{S_n - np}{\sqrt{n\sigma^2}} <
	\frac{\varepsilon}{\sqrt{n\sigma^2}}\right\} \approx \\
	\Phi(\frac{\varepsilon}{\sqrt{n\sigma^2}}) - \Phi(-\frac{\varepsilon}{\sqrt{n\sigma^2}}) = \\
	2\Phi_0(\frac{\varepsilon}{\sqrt{n\sigma^2}}) = 2\Phi_0(\frac{40}{\sqrt{400 \times 0.1275}}) =
	2\Phi_0(5.6) \approx 0.49
\end{gather*}

\end{document}

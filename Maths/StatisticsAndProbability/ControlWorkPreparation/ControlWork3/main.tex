\documentclass[12pt]{article}
\usepackage[T1, T2A]{fontenc}
\usepackage[utf8]{inputenc}
\usepackage[russian]{babel}
\usepackage{hyperref}
\usepackage{datetime}
\usepackage{amsmath}
\usepackage{amsfonts}
\usepackage{tikz}
\graphicspath{ {./Images/} }

% For slope Fields
\usepackage{pgfplots}
\pgfplotsset{compat=newest}
\usepgfplotslibrary{fillbetween}
\usetikzlibrary{calc}
\usetikzlibrary{shapes.geometric, arrows.meta}
\pgfplotsset{compat=1.8}

\author{Григорий Матюхин}
\date{\today}
\title{
	Теория вероятностей и математическая статистика \\
	\large Подготовка к контрольной работе \textnumero3.
}

\DeclareMathOperator{\cov}{cov}

\begin{document}
\maketitle
\newpage
\tableofcontents
\newpage

\section{Пробный вариант}
\subsection{Задание \textnumero1.}
Дискретная двумерная случайная величина $(\xi, \eta)$ задана рядом распределения. \\
\begin{tabular}{|c|c|c|c|}
	\hline
	$\xi \textbackslash \mu$ & $-1$   & $0$    & $1$    \\
	\hline
	$-1$                     & $0.12$ & $0.14$ & $0.21$ \\
	\hline
	$2$                      & $0.25$ & $0.15$ & $0.13$ \\
	\hline
\end{tabular} \\
Найдите:
\begin{enumerate}
	\item ряды распределения случайных величин $\xi$ и $\eta$;
	\item значение совместной функции распределения $F(\xi, \eta)$ в точке $(1;0,5)$;
	\item условное распределение случайной величины $\eta$ при условии $\xi$;
	\item математическое ожидание и дисперсию случайной величины $\mu = |\eta| \times \xi$;
	\item ковариацию и коэффициент корреляции случайных величин $\xi$ и $\eta$.
\end{enumerate}

\subsubsection*{Решение}
\begin{enumerate}
	\item \mbox{}\\
	      % sum of probabilities in rows
	      \begin{tabular}{|c|c|c|}
		      \hline
		      $\xi $ & $-1$   & $2$    \\
		      \hline
		      $p$    & $0.47$ & $0.53$ \\
		      \hline
	      \end{tabular}
	      % sum of probabilities in columns
	      \begin{tabular}{|c|c|c|c|}
		      \hline
		      $\eta $ & $-1$   & $0$    & $1$    \\
		      \hline
		      $p$     & $0.37$ & $0.29$ & $0.34$ \\
		      \hline
	      \end{tabular}

	\item \mbox{}\\
	      \begin{equation*}
		      F(x, y) = P\{\xi < x, \eta < y\} =
		      \begin{cases}
			      \begin{aligned}[t]
				      0,    & \quad x \leq -1, y \leq -1         \\
				      0.12, & \quad -1 < x \leq 0, -1 < y \leq 0 \\
				      0.26, & \quad -1 < x \leq 0, -0 < y \leq 1 \\
				      0.47, & \quad -1 < x \leq 0, y > 1         \\
				      0.36, & \quad x > 2, -1 < y \leq 0         \\
				      0.66, & \quad x > 2, -0 < y \leq 1         \\
				      1,    & \quad x > 2, y > 1                 \\
			      \end{aligned}
		      \end{cases}
	      \end{equation*}
	      \begin{equation*}
		      F(1, 0.5) = P\{\xi < 1, \eta < 0.5\} = 0.26
	      \end{equation*}

	\item \mbox{}\\
	      \begin{equation*}
		      P(\xi = i|\eta = j) = \frac{P(\xi = j|\eta = i)}{P(\xi = j)}
	      \end{equation*}
	      % \eta `if` \xi => divide by corresponding \xi from the first step
	      \begin{tabular}{|c|c|c|c|}
		      \hline
		      $\xi \textbackslash \eta$ & $-1$   & $0$    & $1$    \\
		      \hline
		      $-1$                      & $0.25$ & $0.29$ & $0.51$ \\
		      \hline
		      $2$                       & $0.47$ & $0.28$ & $0.24$ \\
		      \hline
	      \end{tabular} \\

	\item \mbox{}\\
	      \[\mu = g(\xi, \eta) = |\eta| \times \xi\]
	      \begin{tabular}{|c|c|c|c|c|c|c|}
		      \hline
		      $\mu$ & $g(-1, -1) = -1$ & $g(-1, 0) = 0$ & $g(-1, 1) = -1$ & $g(2, -1) = 2$ & $g(2, 0) = 0$ & $g(2, 1) = 2$ \\
		      \hline
		            & $0.12$           & $0.14$         & $0.21$          & $0.25$         & $0.15$        & $0.13$        \\
		      \hline
	      \end{tabular} \\
	      \begin{tabular}{|c|c|c|c|}
		      \hline
		      $\mu$ & $-1$   & $0$    & $2$    \\
		      \hline
		            & $0.33$ & $0.29$ & $0.38$ \\
		      \hline
	      \end{tabular} \\
	      \begin{gather*}
		      M\mu = \sum_iX_ip_i = -1 \times 0.33 + 0 \times 0.29 + 2 \times 0.38 = 0.43 \\
		      D\mu = M(\mu - M\mu)^2 = \sum_i\left(X_i-M\mu\right)^2p_i = \\
		      = (-1 - 0.43)^2 \times 0.33 + (0 - 0.43)^2 \times 0.29 + (2 - 0.43)^2 \times 0.38 = \\
		      = 0.67 + 0.03 + 0.94 = 1.67
	      \end{gather*}

	\item \mbox{}\\
	      \begin{gather*}
		      M\xi = -1 \times 0.47 + 2 \times 0.53 = 0.59 \\
		      M\eta = -1 \times 0.37 + 0 \times 0.29 + 1 \times 0.34 = -0.03 \\
		      \cov(\xi, \eta) = M\mathring{\xi}\mathring{\eta} = M\left[\left(\xi - M\xi\right)\left(\eta - M\eta\right)\right] = \\
		      = \sum_{i, j}\left(X_i - M\xi\right)\left(Y_j - M\eta\right)p_{i,j} = \\
		      = -0.31 \\ % Todo: check python code again
		      D\xi = \sum_i\left(X_i-M\xi\right)^2p_i = \\
		      = (-1 - 0.59)^2 \times 0.47 + (2 - 0.59)^2 \times 0.53 = \\
		      = 2.24 \\
		      D\eta = \sum_i\left(X_i-M\eta\right)^2p_i = \\
		      = (-1 - (-0.03))^2 \times 0.37 + (0 - (-0.03))^2 \times 0.29 + (1 - (-0.03))^2 \times 0.34 = \\
		      = 0.7 \\
		      \rho = \frac{\cov(\xi, \eta)}{\sqrt{D\xi \times D\eta}} = \\
		      \frac{-0.31}{\sqrt{2.24 \times 0.7}} = -0.2
	      \end{gather*}
\end{enumerate}

\subsection{Задание \textnumero2.}
Задана плотность совместного распределения непрерывной двумерной случайной величины $(\xi, \eta)$:
\begin{equation*}
	p_{\xi\eta} =
	\begin{cases}
		\begin{aligned}[t]
			0,  & \quad (x;y) \notin D, \\
			Cy, & \quad (x;y) \in D
		\end{aligned}
	\end{cases}
\end{equation*}
где область $D$ --- треугольник с вершинами в точках $(0;-1)$, $(1;0)$ и $(-1;0)$.
Найдите
\begin{enumerate}
	\item значение постоянной $C$;
	\item частные плотности распределения случайных величин $\xi$ и $\eta$;
	\item вероятность попадания случайной величины $(\xi, \eta)$ в прямоугольник с вершинами в точках $(-0.5;1)$, $(0;1)$, $(0;-1)$ и $(-0.5;-1)$;
	\item значение совместной функции распределения $F_{\xi\eta}(x,y)$ в точке $(1;-0.5)$;
	\item математические ожидания и дисперсии с. в. $\xi$ и $\eta$;
	\item ковариацию и коэффициент корреляции с.в. $\xi$ и $\eta$;
\end{enumerate}

\subsubsection*{Решение}
\begin{enumerate}
	\item \mbox{}\\
	      \begin{tikzpicture}
		      \begin{axis}[
				      axis lines = middle,
				      xlabel = {$x$},
				      ylabel = {$y$},
				      xmin=-2, xmax=2,
				      ymin=-2, ymax=2]

			      % Triangle 
			      % Top
			      \addplot [name path = T,
				      domain = -1:1] {0}
			      node [pos=1, above] {};

			      % Bottom right
			      \addplot [name path = Br,
				      domain = 0:1] {x - 1}
			      node [pos=1, above] {};

			      % Bottom left
			      \addplot [name path = Bl,
				      domain = -1:0] {- x - 1}
			      node [pos=1, above] {};

			      \path[%draw=blue,thick,->,
				      intersection segments={
						      of=Bl and Br,
						      % sequence={R1--L1[reverse]}
					      }, name path=B];

			      \path[%draw=red,thick,->,
				      fill=orange,
				      intersection segments={
						      of=T and B,
						      % sequence={L2--R2}
					      }];

			      \path[name intersections={of=Br and Bl,by=N}];
			      \draw (N) node[right=.5em] {A} circle[radius=2pt];

			      \path[name intersections={of=T and Br,by=M}];
			      \draw (M) node[above=.5em] {B} circle[radius=2pt];

			      \path[name intersections={of=T and Bl,by=M}];
			      \draw (M) node[above=.5em] {C} circle[radius=2pt];

		      \end{axis}
	      \end{tikzpicture}

	      Из условия нормировки
	      \begin{equation*}
		      \int_{-\infty}^{\infty}\int_{-\infty}^{\infty}p(x,y)dxdy = 1 \Rightarrow \iint_{D}p(x,y)dxdy = 1
	      \end{equation*}
	      Получаем
	      \begin{gather*}
		      \iint_{D}p(x,y)dxdy = \iint_{D}Cydxdy = C\iint_{D}ydxdy = \\
		      C \left( \int_{-1}^{0}\left(\int_{-x-1}^{0}ydy\right)dx + \int_{0}^{1}\left(\int_{x-1}^{0}ydy\right)dx \right) = \\
		      \cdots \\
		      = C \left(-\frac{1}{6} + \frac{7}{6}\right) = C \Rightarrow C = 1 \\
		      % Note: C is actually -3 here,
		      % I have no idea how did I calculate this so wrong,
		      % this is the correct way
		      % C \int_{-1}^{0}\int_{-y-1}^{y+1}ydxdy = -\frac{1}{3}C => C = -3
	      \end{gather*}
	      \begin{equation*}
		      p_{\xi\eta} =
		      \begin{cases}
			      \begin{aligned}[t]
				      0, & \quad (x;y) \notin D, \\
				      y, & \quad (x;y) \in D
			      \end{aligned}
		      \end{cases}
	      \end{equation*}

	\item \mbox{}\\
	      \begin{gather*}
		      p_\xi(x) = \int_{-\infty}^{\infty}p(x,y)dy = \\
		      = \int_{-x-1}^{0}ydy + \int_{x-1}^{0}ydy = \\
		      \cdots \\
		      -(x^2 + 1), \quad x \in [-1; 1]
	      \end{gather*}

	      \begin{gather*}
		      p_\eta(y) = \int_{-\infty}^{\infty}p(x,y)dx = \\
		      = \int_{-y-1}^{y+1}ydx = \\
		      \cdots \\
		      = 2y(y + 1), \quad y \in [-1; 0]
	      \end{gather*}

	\item \mbox{}\\
	      \begin{gather*}
		      P\{-0.5 < \xi < 0, -1 < \eta < 1\} = \int_{-0.5}^{0}dx\int_{-1}^{1}p(x,y)dy = \\
		      % \int_{-0.5}^{0}dx\int_{-1}^{1}ydy % partially otside of the triangle
		      % \int_{-y-1}^{0}dx\int_{-1}^{0}ydy % integralls in the wrong order
		      % \int_{-1}^{0}dy\int_{-y-1}^{0}ydx = -\frac{1}{6} \\
		      \int_{-0.5}^{0}dx\int_{-x-1}^{0}ydy \approx -0.14 %is more believable
	      \end{gather*}

	\item \mbox{}\\
	      % either 0 or
	      \begin{gather*}
		      F(1,-0.5) = \int_{-\infty}^{1}dx\int_{-\infty}^{-0.5}p(x,y)dy = \\
		      % \int_{0}^{1}dx\int_{y+1}^{-0.5}ydy = \\
		      % \int_{x-1}^{1}dx\int_{y+1}^{-0.5}ydy = \\
		      \int_{0}^{-0.5}dy\int_{y+1}^{1}ydx = \\
		      \cdots \\
		      \approx 0.041
	      \end{gather*}

	\item \mbox{}\\
	      \begin{gather*}
		      M\xi = \int_{-\infty}^{\infty}xp(x)dx = \\
		      \int_{-\infty}^{\infty}-x(x^2 + 1)dx = \\
		      \int_{-1}^{1}-x(x^2+1)dx = 0 \\
		      \\
		      D\xi = M(\xi - M\xi)^2 = \int_{-\infty}^{\infty}(x - M\xi)^2p(x)dx = \\
		      \int_{-1}^{1}-(x - 0)^2(x^2 + 1)dx = \\
		      \int_{-1}^{1}-x^2(x^2+1)dx = -\frac{16}{15}\\
	      \end{gather*}

	      \begin{gather*}
		      M\eta = \int_{-\infty}^{\infty}yp(y)dy = \\
		      \int_{-\infty}^{\infty}y2y(y + 1)dy = \\
		      \int_{-1}^{0}y2y(y+1)dy = \frac{1}{6} \\
		      \\
		      D\xi = M(\eta - M\eta)^2 = \int_{-\infty}^{\infty}(y - M\xi)^2p(y)dy = \\
		      \int_{-1}^{0}(y - \frac{1}{6})^2 2y(y + 1) dy = -\frac{89}{540}
	      \end{gather*}

	\item \mbox{}\\
	      \begin{gather*}
		      \cov(\xi, \eta) = \int_{-\infty}^{\infty}\int_{-\infty}^{\infty}(x - M\xi)(y - M\eta)p(x, y)dxdy = \\
		      \iint_{D}(x - M\xi)(y - M\eta)p(x, y)dxdy = \\
		      \iint_{D}(x - 0)(y - \frac{1}{6})ydxdy = \\
		      \iint_{D}x(y - \frac{1}{6})ydxdy = \\
		      \int_{-1}^{0}(y - \frac{1}{6})ydy\int_{-y-1}^{y+1}dx = \frac{2}{9}\\
	      \end{gather*}

	      \begin{gather*}
		      \rho = \frac{\cov(\xi, \eta)}{\sqrt{D\xi \times D\eta}} = \\
		      \frac{\frac{2}{9}}{\sqrt{-\frac{16}{15} \times -\frac{89}{540}}} = \\
		      \frac{5\sqrt{89}}{89}
	      \end{gather*}

\end{enumerate}

\end{document}

\section{Домашняя работа}

\subsection{Задача на оценку}
\subsubsection{Условие}
$y' = -6x^2 + 6, y(0) = 1$

\subsubsection{Поиск решения графически}
Найдем углы для поля направлений \\
Возьмем $x \in \{-4, -3.5, ..., 4\}$ \\
Для каждого $x$ надем $\alpha$ из $\tg \alpha = -6x^2 + 6$: \\
$x = \pm 4 \Rightarrow \alpha = \pi n - \arctan(90), n \in \mathbb{Z}$ \\
$x = \pm3.5 \Rightarrow \alpha = \pi n - \arctan\left(\frac{135}{2}\right), n \in \mathbb{Z}$ \\
$x = \pm 3 \Rightarrow \alpha = \pi n - \arctan(48), n \in \mathbb{Z}$ \\
$x = \pm 2.5 \Rightarrow \alpha = \pi n - \arctan\left(\frac{63}{2}\right), n \in \mathbb{Z}$ \\
$x = \pm 2 \Rightarrow \alpha = \pi n - \arctan(18), n \in \mathbb{Z}$ \\
$x = \pm 1.5 \Rightarrow \alpha = \pi n - \arctan\left(\frac{15}{2}\right), n \in \mathbb{Z}$ \\
$x = \pm 1 \Rightarrow \alpha = \pi n , n \in \mathbb{Z}$ \\
$x = \pm 0.5 \Rightarrow \alpha = \pi n - \arctan\left(\frac{9}{2}\right), n \in \mathbb{Z}$ \\
$x = 0 \Rightarrow \alpha = \pi n + \arctan(6), n \in \mathbb{Z}$ \\

\newpage
Построим поле направлений и приближенные интегральные кривые: \\
% \vspace{5mm}
\begin{tikzpicture}[declare function={f(\x,\y)=-6 * \x * \x + 6;}]
	\def\xmax{5} \def\xmin{-5}
	\def\ymax{5} \def\ymin{-5}
	\def\nx{19}
	\def\ny{19}

	% square grid
	\draw[very thin,color=gray] (\xmin,\ymin) grid (\xmax, \ymax);

	\pgfmathsetmacro{\hx}{(\xmax-\xmin)/\nx}
	\pgfmathsetmacro{\hy}{(\ymax-\ymin)/\ny}
	\pgfmathsetmacro{\nyl}{\ny-1}
	\pgfmathsetmacro{\nxl}{\ny-1}

	% slope lines
	\foreach \i in {1,...,\nxl}
		{
			\foreach \j in {1,...,\nyl}
				{
					\pgfmathsetmacro{\yprime}{f({\xmin+\i*\hx},{\ymin+\j*\hy})}
					\draw[thick,orange,shift={({\xmin+\i*\hx},{\ymin+\j*\hy})}]
					(0,0)--($(0,0)!2mm!(.1,.1*\yprime)$);
				}
		}

	% axis ticks
	\foreach \i in {\xmin,...,\xmax}
	\draw(\i cm,1pt + \ymin cm) -- (\i cm,-1pt + \ymin cm) node[anchor=north] {$\i$};
	\foreach \j in {\ymin,...,\ymax}
	\draw(1pt + \xmin cm, \j cm) -- (-1pt + \xmin cm, \j cm) node[anchor=east] {$\j$};

	% axis names
	\draw(current bounding box.south) node[below right] {$x$};
	\draw(current bounding box.west) node[above left] {$y$};

	% plot name
	\draw (current bounding box.south) node[below]
	{Приближенные интегральные кривые для \quad $y'=-6x^2 + 6$.};

	% several possible solutions
	% draw solution lines at the very end because we use clip
	\clip (\xmin, \ymin) rectangle (\xmax, \ymax);
	\foreach \i/\c in {-2/red, 0/green, 2/blue}
		{
			\draw[\c,very thick] plot[domain=\xmin:\xmax, smooth] (\x,{\i -2 * \x ^ 3  + 6 * \x});
		}
\end{tikzpicture}

\subsubsection{Поиск решения аналитически}
\begin{gather*}
	y' = -6x^2 + 6 \\
	y = \int \left(-6x^2 + 6\right)dx \\
	y = 6\int \left(-x^2 + 1\right)dx \\
	y = 6 \left( \int-x^2dx + \int1dx \right) \\
	y = 6 \left( -\frac{x^3}{3} + x + C \right) \\
	y = -2x^3 + 6x + 6C \\
\end{gather*}

Так как $C$ -- некая константа, заменим $6C = C$:
\[y = C - 2x^3 + 6x\]

Найдем $C$ при $y(0) = 1$:
\begin{gather*}
	y = C - 2x^3 + 6x, y(0) = 1 \\
	1 = C - 2 \cdot 0^3 + 6 \cdot 0 \\
	1 = C - 2 \cdot 0^3 + 6 \cdot 0 \\
	C = 1 \\
\end{gather*}

Тогда частное решение:
\[y = 1 - 2x^3 + 6x\]

Начертим:

\vspace{5mm}
\begin{tikzpicture}
	\def\xmax{5} \def\xmin{-5}
	\def\ymax{5} \def\ymin{-5}
	\def\nx{19}
	\def\ny{19}

	% square grid
	\draw[very thin,color=gray] (\xmin,\ymin) grid (\xmax, \ymax);

	% axis ticks
	\foreach \i in {\xmin,...,\xmax}
	\draw(\i cm,1pt + \ymin cm) -- (\i cm,-1pt + \ymin cm) node[anchor=north] {$\i$};
	\foreach \j in {\ymin,...,\ymax}
	\draw(1pt + \xmin cm, \j cm) -- (-1pt + \xmin cm, \j cm) node[anchor=east] {$\j$};

	% axis names
	\draw(current bounding box.south) node[below right] {$x$};
	\draw(current bounding box.west) node[above left] {$y$};

	% plot name
	\draw (current bounding box.south) node[below]
	{Решение $y' = -6x^2 + 6$ при $y(0) = 1$ с точками экстремума};

	% several possible solutions
	% draw solution lines at the very end because we use clip
	\clip (\xmin, \ymin) rectangle (\xmax, \ymax);
	\draw[blue,very thick] plot[domain=\xmin:\xmax, smooth] (\x,{1 - 2 * \x ^ 3  + 6 * \x});

	% interesting points
	\draw(-1 cm, -3 cm) node[below] {(-1, -3)};
	\draw(1 cm, 5 cm) node[below left] {(1, 5)};
\end{tikzpicture}

Точки $A(-1; -3)$ и $B(1; 5)$ -- точки перегиба. \\
Данное частное решение совпадает с указанными выше полем направлений и интегралтными кривыми.

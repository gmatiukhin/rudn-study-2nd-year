\documentclass[12pt]{article}
\usepackage[T1, T2A]{fontenc}
\usepackage[utf8]{inputenc}
\usepackage[russian]{babel}
\usepackage{hyperref}
\usepackage{datetime}
\usepackage{amsmath}
\usepackage{amsfonts}
\usepackage{tikz}

% For slope Fields
\usepackage{pgfplots}
\pgfplotsset{compat=newest}
\usepgfplotslibrary{fillbetween}
\usetikzlibrary{calc, quotes, positioning, decorations.text}
\usetikzlibrary{shapes.geometric}

\usepackage[14pt]{extsizes} % для того чтобы задать нестандартный 14-ый размер шрифта
\usepackage[utf8]{inputenc}
\usepackage[russian]{babel} % поддержка русского языка
\usepackage{amsmath}  %  математические символы
\usepackage[left=20mm, top=15mm, right=15mm, bottom=30mm, footskip=15mm]{geometry} % настройки полей документа
\usepackage{indentfirst} % по умалчанию убирается отступ у первого абзаца в секции, это отменяет это.
\usepackage{paralist} % добавить компактные списки (compactitem, compactenum, compactdesc)

\usepackage{fancyvrb}
\usepackage{framed}

\usepackage{float}
\floatstyle{ruled}
\newfloat{termfloat}{thp}{lop}
\floatname{termfloat}{Снимок экрана}



\begin{document}
% \maketitle
\begin{titlepage}

	\begin{center}
		\hfill \break
		\textbf{
			\large{РОССИЙСКИЙ УНИВЕРСИТЕТ ДРУЖБЫ НАРОДОВ}\\
			\normalsize{Факультет физико-математических и естественных наук}\\
			\normalsize{Кафедра прикладной информатики и теории вероятностей}\\
		}
		\vspace*{\fill}
		\Large{\textbf{Домашнее задание \textnumero 2}}
		\\
		\underline{\textit{\normalsize{Дисциплина: Дифференциальные уравнения}}}
		\vspace*{\fill}

	\end{center}

	\begin{flushright}
		Студент: \underline{Григорий Матюхин}\\ \vspace{0.5cm}
		Группа: \underline{НПИбд-01-21}
	\end{flushright}


	\begin{center} \textbf{МОСКВА} \\ 2022 г. \end{center}
	\thispagestyle{empty} % выключаем отображение номера для этой страницы

\end{titlepage}
\newpage
\tableofcontents
\newpage

\section{Задача на оценку}
\subsection*{Условие}
$y' = y - 6, y(0) = 1$

\subsection{Построим график зависимости производной}
\begin{tikzpicture}
	\begin{axis}[
			axis lines = middle,
			xlabel = {$y$},
			ylabel = {$y'$},
			% xtick distance = 1,
			% ytick distance = 1,
			xmin=-3, xmax=10,
			ymin=-10, ymax=3]

		\addplot [
			color = blue,
			domain = -10:10,
		] {x - 6}
		node [midway, below right] {$y' = y - 6$};
	\end{axis}
\end{tikzpicture}

\subsection{Построим поле направлений и интегральные кривые}
Найдем углы для поля направлений.
Возьмем $y \in \{-9.5, -9, -8.5 ..., 9.5\}$.
Для каждого $y$ надем $\alpha$ из $\tg \alpha = y - 6$ и начертим. \\
\begin{tikzpicture}[declare function={f(\x,\y)=\y - 6;}, scale = 0.4]
	\def\xmax{10} \def\xmin{-10}
	\def\ymax{10} \def\ymin{-10}
	\def\nx{20}
	\def\ny{20}

	% square grid
	\draw[very thin,color=gray, step=2 cm] (\xmin, \ymin) grid (\xmax, \ymax);

	\pgfmathsetmacro{\hx}{(\xmax-\xmin)/\nx}
	\pgfmathsetmacro{\hy}{(\ymax-\ymin)/\ny}
	\pgfmathsetmacro{\nyl}{\ny-1}
	\pgfmathsetmacro{\nxl}{\ny-1}

	% slope lines
	\foreach \i in {1,...,\nxl}
		{
			\foreach \j in {1,...,\nyl}
				{
					\pgfmathsetmacro {\yprime} {f({\xmin + \i * \hx}, {\ymin + \j *\hy})}
					\draw[thick,orange,shift={({\xmin + \i * \hx}, {\ymin + \j * \hy})}]
					(0, 0) -- ($(0, 0) !6mm! (0.1, 0.1 * \yprime)$);
				}
		}

	% axis ticks
	% \def\hxmax{5} \def\hxmin{-5}
	% \def\hymax{5} \def\hymin{-5}
	\foreach \i in {\xmin,...,\xmax}
		{
			% \pgfmathsetmacro{\half}{\i / 2};
			\pgfmathparse{int(mod(\i,2))};
			\ifnum0=\pgfmathresult\relax
				\draw (\i cm,\ymin cm) node [anchor=north] {$\i$};
			\fi
			% \node[anchor=north] at (\half cm,\hymin cm) {$\i$};
		}
	\foreach \j in {\ymin,...,\ymax}
		{
			% \pgfmathsetmacro{\half}{\j / 2};
			\pgfmathparse{int(mod(\j,2))};
			\ifnum0=\pgfmathresult\relax
				\draw (\xmin cm, \j) node [anchor=east] {$\j$};
			\fi
		}

	% axis names
	\draw(current bounding box.south) node[below right] {$x$};
	\draw(current bounding box.west) node[above left] {$y$};

	% % plot name
	% \draw (current bounding box.south) node[below]
	% {Приближенные интегральные кривые для \quad $y'=-6x^2 + 6$.};

	% several possible solutions
	% draw solution lines at the very end because we use clip
	\clip (\xmin, \ymin) rectangle (\xmax, \ymax);
	\draw[green,very thick] plot[domain=\xmin:\xmax, smooth] (\x, {6});
	\foreach \i/\c in {-2/red, -1/violet, 1/magenta, 2/blue}
		{
			\draw[\c,very thick] plot[domain=\xmin:3, smooth] (\x, {\i * exp(\x) + 6});
		}
\end{tikzpicture}

\subsection{Решим аналитически}
\subsubsection{Найдем общее решение}
$y' = y - 6 \Leftrightarrow y' - y = -6$ --- ЛНУ 1-го порядка \\
Решим соответствующее ЛОУ: $v' - v = 0$
\begin{gather*}
	v' - v = 0 \\
	v' = v \\
	\frac{dv}{dx} = v \\
	\frac{dv}{v} = dx \\
	\int\frac{dv}{v} = \int dx \\
	\ln{v} = x + C_1 \\
	e^{\ln{v}} = e^{x + C_1} \\
	v = e^xe^{C_1}
\end{gather*}
Упростим константу и получим общее решение ЛОУ
\begin{equation} \label{eq:found_v}
	v = C_1e^x \Leftrightarrow v = C_1v_1 \Leftrightarrow y_\textup{о.о.} = C_1e^x
\end{equation}
Перейдем к решению непосредственно самого неоднородного уравнения методом вариации постоянной.
Будем искать решение в виде произведения функций
\begin{equation} \label{eq:y_temp}
	y = u(x)v_1 \Leftrightarrow y = u(x)e^x
\end{equation}
Подставим в исходное неоднородное уравнение и надем $u(x)$
\begin{gather*}
	(u(x)e^x)' = u(x)e^x - 6 \\
	u(x)'e^x + u(x)(e^x)' = u(x)e^x - 6 \\
	u(x)'e^x + u(x)e^x = u(x)e^x - 6 \\
	u(x)'e^x = - 6 \\
	u(x)' = -\frac{6}{e^x} \\
\end{gather*}
\begin{gather*}
	u(x) = \int-\frac{6}{e^x}dx \\
	u(x) = -6\int e^{-x}dx \\
	u(x) = -6 (-e^{-x} + C_2) \\
	u(x) = 6e^{-x} - 6C_2
\end{gather*}
Упростим константу
\begin{equation} \label{eq:found_u}
	u(x) = 6e^{-x} + C_2 \\
\end{equation}
Подставим в \ref{eq:y_temp}
\begin{gather*}
	y = (6e^{-x} + C_2)e^x \\
	y = 6e^{-x}e^x + C_2e^x \\
	y = 6 + C_2e^x
\end{gather*}
Положим $C_2 = 0$, получим частное решение ЛНУ
\begin{equation} \label{eq:nle}
	y_\textup{ч.н.} = 6
\end{equation}
Найдем общее решение ЛНУ как сумму частного решения ЛНУ и общего решения ЛОУ
\begin{equation}
	y_\textup{о.н.} = C_1e^x + 6
\end{equation}

\subsubsection{Найдем частное решение при заданном условии}
\begin{gather*}
	y_\textup{о.н.} = C_1e^x + 6, y(0) = 1 \\
	1 = C_1e^0 + 6 \\
	C_1 = -5
\end{gather*}
Получили
\begin{equation*}
	y_\textup{ч.н.} = -5e^x + 6
\end{equation*}

\subsubsection{Построим график}

\begin{tikzpicture}
	\begin{axis}[
			axis lines = middle,
			xlabel = {$x$},
			ylabel = {$y$},
			% xtick distance = 1,
			% ytick distance = 1,
			xmin=-3, xmax=5,
			ymin=-10, ymax=10]

		\addplot [
			color = blue,
			domain = -10:1.25,
			samples = 100
		] {-5 * e^x + 6}
		node [midway, right] {$y=-5e^x + 6$};
		% \draw[decoration={
		% 			text along path,
		% 			text={i$y=-5e^x + 6$},
		% 			text align={center},
		% 			raise=0.2cm},decorate] (a) to (b);
	\end{axis}
\end{tikzpicture}

\subsection{Найдем стационарные решения}

\begin{gather*}
	y' = 0 \\
	y - 6 = 0 \\
	y = 6
\end{gather*}

\subsubsection{Начертим стационарные решения}
\begin{tikzpicture}
	\begin{axis}[
			axis lines = middle,
			xlabel = {$x$},
			ylabel = {$y$},
			% xtick distance = 1,
			% ytick distance = 1,
			xmin=-3, xmax=5,
			ymin=-10, ymax=10]

		\addplot [
			color = blue,
			domain = -10:1.25,
			samples = 100
		] {-5 * e^x + 6}
		node [midway, right] {$y=-5e^x + 6$};

		\addplot [
			color = red,
			domain = -10:10,
			samples = 100
		] {6}
		node [midway, right] {};
	\end{axis}
\end{tikzpicture}

\end{document}

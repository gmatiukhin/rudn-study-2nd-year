\section{Лекция}

\subsection{Задача \textnumero1.1}
\paragraph{Условие}
Точка движется по оси $x$ со скоростью $v = 3$.\\
Найти возможные законы ее движения. Определить среди этих законов тот, для которого $x = 0$ при $t = 0$, а также тот, для которого $x = 1$ при $t = 1$. 
\paragraph{Решение}
\[x = x(t)\]
\[x'(t) = v(t)\]
\begin{equation}
  x(t) = \int v(t)dt + C
  \label{eq:dot_movement}
\end{equation}
Получили уравнение возможных законов движения для точки.
\paragraph{Если $v = 3$:}
\[x(t) = \int 3dt + C\]
\[x(t) = 3\int dt + C\]
\begin{equation}
  x(t) = 3t + C
  \label{eq:vel_3}
\end{equation}
Из \ref{eq:vel_3}, $x = 0$ при $t = 0$:
\[0 = 3 \cdot 0 + C\]
\[0 = 0 + C\]
\[C = 0\]
\[x(t) = 3t\]
Из \ref{eq:vel_3}, $x = 1$ при $t = 1$:
\[1 = 3 \cdot 1 + C\]
\[1 = 3 + C\]
\[C = -2\]
\[x(t) = 3t - 2\]
\paragraph{Если $v = 2t$:}
Из \ref{eq:dot_movement}:
\[x(t) = \int 2t dt + C\]
\[x(t) = 2 \int t dt + C\]
\[x(t) = 2 \frac{t^2}{2} + C\]
\begin{equation}
  x(t) = t^2 + C
  \label{eq:suqared_vel}
\end{equation}
Из \ref{eq:suqared_vel}, $x = 0$ при $t = 0$:
\[0 = 0^2 + C\]
\[0 = 0 + C\]
\[C = 0\]
\[x(t) = t^2\]
Из \ref{eq:suqared_vel}, $x = 1$ при $t = 1$:
\[1 = 1^2 + C\]
\[1 = 1 + C\]
\[C = 0\]
\[x(t) = t^2\]

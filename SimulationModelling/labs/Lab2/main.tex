\documentclass[12pt]{article}
\usepackage[T1, T2A]{fontenc}
\usepackage[utf8]{inputenc}
\usepackage[russian]{babel}
\usepackage{hyperref}
\usepackage{datetime}
\usepackage{listings}
\usepackage{amsmath}
\usepackage{amsfonts}
\usepackage{tikz}
\graphicspath{ {./Images/} }

% For slope Fields
\usepackage{pgfplots}
\usetikzlibrary{calc}
\usetikzlibrary{shapes.geometric, arrows.meta, positioning}
\pgfplotsset{compat=1.8}

\usepackage[14pt]{extsizes} % для того чтобы задать нестандартный 14-ый размер шрифта
\usepackage[utf8]{inputenc}
\usepackage[russian]{babel} % поддержка русского языка
\usepackage{amsmath}  %  математические символы
\usepackage[left=20mm, top=15mm, right=15mm, bottom=30mm, footskip=15mm]{geometry} % настройки полей документа
\usepackage{indentfirst} % по умалчанию убирается отступ у первого абзаца в секции, это отменяет это.
\usepackage{paralist} % добавить компактные списки (compactitem, compactenum, compactdesc)

\usepackage{fancyvrb}
\usepackage{framed}

\DeclareMathOperator{\cov}{cov}

\begin{document}
% \maketitle
\begin{titlepage}

	\begin{center}
		\hfill \break
		\textbf{
			\large{РОССИЙСКИЙ УНИВЕРСИТЕТ ДРУЖБЫ НАРОДОВ}\\
			\normalsize{Факультет физико-математических и естественных наук}\\
			\normalsize{Кафедра прикладной информатики и теории вероятностей}\\
		}
		\vspace*{\fill}
		\Large{\textbf{Лабораторная работа №2}}
		\\
		\underline{\textit{\normalsize{Дисциплина: Имитационное моделирование}}}
		\vspace*{\fill}

	\end{center}

	\begin{flushright}
		Студент: \underline{Григорий Матюхин}\\ \vspace{0.5cm}
		Группа: \underline{НПИбд-01-21}
	\end{flushright}


	\begin{center} \textbf{МОСКВА} \\ 2023 г. \end{center}
	\thispagestyle{empty} % выключаем отображение номера для этой страницы

\end{titlepage}
\newpage
\tableofcontents
\newpage

\section{Постановка задания}
Смоделировать СеМО, состоящую из $k$ узлов, расставленных по кругу.
Каждый узел состоит из накопителя и прибора.
В начальный момент времени в каждом узле имеется $m$ заявок.
Обслуживание заявок является экспоненциальным, параметры обслуживания различны.
Извне на сеть поступает два пуассоновских потока заявок: положительных и отрицательных.
При этом интенсивность потока положительных заявок в два раза ниже, чем отрицательных.
И те, и другие заявки с равной вероятностью выбирают один из узлов сети.
При этом положительные заявки пополняют количество заявок,
присутствовавших в сети в начальный момент времени и ничем от них не отличаются.
Отрицательные заявки, попадая в узел,
убивают обслуживаемую заявку (если такая имеется) и сразу покидают сеть.

Все параметры модели должны задаваться с помощью сохраняемых величин.
Числовые значения параметров придумать самостоятельно.
Определить время, через которое в сети не останется ни одной полезной заявки.
Построить график зависимости этого времени от параметра $m$.

\section{Код программы}
\begin{lstlisting}
; Constants
; Nodes
NodeInit  EQU 10

Node1L    EQU 37
Node1S    EQU 5
Node2L    EQU 30
Node2S    EQU 4
Node3L    EQU 35
Node3S    EQU 7
; Streams
posL      EQU 6
negL      EQU (posL/2)


; Input streams

positive  GENERATE (POISSON(1,posL))
          SAVEVALUE pos+,1
          SAVEVALUE tasks+,1
          ASSIGN type,1
          TRANSFER PICK,n_1,n_3

negative  GENERATE (POISSON(1,negL))
          SAVEVALUE neg+,1
          ASSIGN type,-1
          TRANSFER PICK,n_1,n_3


; Put these nodes together
; so that transfer does not interfer with other transactions,
; because PICK selects in a range
n_1       TRANSFER ,node1
n_2       TRANSFER ,node2
n_3       TRANSFER ,node3


; Nodes

; Node 1
          GENERATE ,,0,NodeInit
          SAVEVALUE tasks+,1
          ASSIGN type,0
node1     QUEUE node1q
  
          TEST E P$type,-1,ctn1
          PREEMPT node1f,,dump,,RE
          RELEASE node1f
          TERMINATE

ctn1      SEIZE node1f
          ADVANCE (EXPONENTIAL(1,Node1L,Node1S))
          RELEASE node1f
          TERMINATE
          TRANSFER ,node2
; end of Node 1

; Node 2
          GENERATE ,,0,NodeInit
          SAVEVALUE tasks+,1
          ASSIGN type,0
node2     QUEUE node2q

          TEST E P$type,-1,ctn2
          PREEMPT node2f,,dump,,RE
          RELEASE node2f
          TERMINATE

ctn2      SEIZE node2f
          ADVANCE (EXPONENTIAL(1,Node2L,Node2S))
          RELEASE node2f
          TRANSFER ,node3
; end of Node 2

; Node 3
          GENERATE ,,0,NodeInit
          SAVEVALUE tasks+,1
          ASSIGN type,0
node3     QUEUE node3q

          TEST E P$type,-1,ctn3
          PREEMPT node3f,,dump,,RE
          RELEASE node3f
          TERMINATE

ctn3      SEIZE node3f
          ADVANCE (EXPONENTIAL(1,Node3L,Node3S))
          RELEASE node3f
          TRANSFER ,node1
; end of Node 3

dump      SAVEVALUE terminated+,1
          TERMINATE

; Clock
          GENERATE 1
          TEST G X$terminated,0
          TEST E X$tasks,X$terminated
          TERMINATE 1

; Control
          START 1
\end{lstlisting}
В данном листинге программы количество узлов $k = 3$,
а изначальное количество заявок на каждом --- $NodeInit = 10$.

\section{Вывод}
Зависимосить времени уничтожения всех полезных заявок от
изначального количества $m$ на каждом из $k = 3$ узлов
\begin{center}
	\begin{tikzpicture}
		\begin{axis}[
				% title={},
				xlabel={Количество заявок на узлах в начальный момент времени},
				ylabel={Время},
				xmin=0, xmax=50,
				ymin=0, ymax=1000,
				% xtick={0,20,40,60,80,100},
				% ytick={0,20,40,60,80,100,120},
				legend pos=north west,
				ymajorgrids=true,
				grid style=dashed,
			]

			\addplot[
				color=blue,
				mark=square,
			]
			coordinates {
					(0,6)(1,102)(2,166)(3,272)(4,153)(5,177)(10,290)(15,344)(20,534)(25,871)(30,727)(35,955)(40,1242)
				};
			% \legend{CuSO\(_4\cdot\)5H\(_2\)O}

		\end{axis}
	\end{tikzpicture}
\end{center}

\end{document}

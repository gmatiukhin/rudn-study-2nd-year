\documentclass[a4page]{article}
\usepackage[14pt]{extsizes} % для того чтобы задать нестандартный 14-ый размер шрифта
\usepackage[utf8]{inputenc}
\usepackage[T1, T2A]{fontenc}
\usepackage[utf8]{inputenc}
\usepackage[russian]{babel} % поддержка русского языка
\usepackage{amsmath}  %  математические символы
\usepackage[left=20mm, top=15mm, right=15mm, bottom=30mm, footskip=15mm]{geometry} % настройки полей документа
\usepackage{indentfirst} % по умалчанию убирается отступ у первого абзаца в секции, это отменяет это.
\usepackage{paralist} % добавить компактные списки (compactitem, compactenum, compactdesc)

\usepackage{fancyvrb}
\usepackage{framed}
\usepackage{url}

\usepackage{float}
\floatstyle{ruled}

\usepackage[
	backend=biber,
	sorting=nyt,
	bibstyle=gost-numeric,
	citestyle=gost-numeric
]{biblatex}

\usepackage[
	bookmarks=true, colorlinks=true, unicode=true,
	urlcolor=black,linkcolor=black, anchorcolor=black,
	citecolor=black, menucolor=black, filecolor=black,
]{hyperref}

\addbibresource{sources.bib}

\renewcommand{\baselinestretch}{1.35}

\begin{document}

\begin{titlepage}
	\begin{center}
		\hfill \break
		\textbf{
			\large{РОССИЙСКИЙ УНИВЕРСИТЕТ ДРУЖБЫ НАРОДОВ}\\
			\normalsize{Факультет физико-математических и естественных наук}\\
			\normalsize{Кафедра прикладной информатики и теории вероятностей}\\
		}
		\vspace*{\fill}
		\Large{\textbf{<<Не геометр да не войдет>>, или как связаны математика, логика и философия?}}
		\\
		\underline{\textit{\normalsize{Итоговое эссе по философии}}}
		\vspace*{\fill}

	\end{center}

	\begin{flushright}
		Студент: \underline{Матюхин Григорий}\\ \vspace{0.5cm}
		Группа: \underline{НПИбд-01-21}
	\end{flushright}

	\begin{center} \textbf{МОСКВА} \\ 2023 г. \end{center}
	\thispagestyle{empty}

\end{titlepage}

\newpage

\tableofcontents

\newpage

\section{Введение}
Вопрос о взаимосвязи математики, логики и философии впервые был задан довольно давно.
Аристотель, Бэкон, Леонардо да Винчи --- многие великие умы человечества занимались этим вопросом.
Это не удивительно: ведь основу взаимодействия философии с какой-либо из наук
составляет потребность использования аппарата философии для проведения исследований в данной области;
математика и логика, несомненно, более всего среди точных наук поддаются философскому анализу.

Цель данного эссе --- показать связь философии, логики и математики,
а также постараться объяснить причину этой связи.
Для простоты исследования рассмотрим связи филосифии с науками по отдельности.

\section{Философия и математика}
История развития науки и культуры в целом свидетельствует о том,
что философия и математика,
как определенные системы знаний возникают примерно в одно время.
С этого момента у них обнаруживается множество точек соприкосновения:
высокий уровень обобщений, единое и многое, конечное и бесконечное,
количество и качество, проблема существования и проблема истины,
противоречие и непротиворечивость, и многое другое.

Кроме того, обе системы теоретического знания характеризуются тем,
зто в них никоим образом не используется практический эксперимент, опыт.
Математики и философы в своих рассуждениях опираются, прежде всего, на <<умозрение>>,
они работают не с материальными, а с абстрактными,
идеальными или идеализированными объектами.

Взаимосвязь математики и философии видна уже в милетской школе философии.
Она проявляется, во-первых, в том, что в математике переходят к доказательствам,
в которых нуждалась и философия;
а во-вторых, в том, что первые милетские философы в математике
<<выходят на путь абстрактных обобщающих построений>>, % TODO: cite
которые превращают математику во всеобще знание.

Влияние античной математики на философию явным образом проявляет себя в пифагорейской школе,
философы которой обнаружили в математике выражение глубинной сущности мира,
нечто связанное с истинной и неизменной природой вещей.
Основной тезис пифагорсизма состоит в том, что <<все есть число>>.

Как подражание числам в пифагорейской школе стали рассматриваться
не только чувственно воспринимаемые вещи, но и их свойства, и отношения.
По аналогии со свойствами того или иного числа или числового соотношения,
их стали трактовать как проявление числовой гармонии.
Можно сказать, что пифагореизм стал первой философской теорией математики.

Связь математики и философии прослеживается и в атомистическом
учении Левкиппа и Демокрита, в котором, в противовес пифагорейцам,
геометрические фигуры считались не умозрительными сущностями,
а материальными телами, состоящими из атомов.
Физическое здесь понималось как логически предшествующее математическому,
умозрительному, абстрактному,
а математические закономерности выступали как вторичные по отношению к атомам.
Атомы же считались и материальной причиной вещей, причиной их существования,
и первыми сущностями, невидимыми простым глазом, а <<зримые>> лишь умом.
В определенной мере можно сказать, что атомизм предвосхитил идеи математического естествознания.

Созданная Демокритом концепция математики, называемая кониепцией математического атомизма,
дала ему возможность решить проблему правомерности теоретических построений математики,
не обращаясь к чувственным образам, как это было у Протагора.

В дальнейшем близость философии и математики обнаруживается у Платона и Аристотеля.

У Платона математика и философия, так же как и у пифагорейцев,
связываются между собой, прежде всего, посредством числа.
В диалоге «Государство» Платон утверждает, что сущность вещи,
скрывающаяся за ее кажимостью, есть математическое.
К истине у Платона ведут именно арифметика и счет, то есть наука о числе.
<<Эта наука, --- отмечает Платон, --- подходит для того,
чтобы установить закон и убедить всех,
кто собирается занять высокие должности в государстве, обратиться к искусству счета>>
и придти <<с помощью самого мышления к созерцанию природы чисел>>
для облегчения <<самой душе ее обращение от становления к истинному бытию>>. % TODO: cite

Платон пошел дальше пифагорейцев в исследованиях реального физического мира,
построенного, с его точки зрения, на основе математических идей.
Этот идеальный мир и был для него истинной реальностью,
к познанию которой он стремился с помощью математики.
Математика для Платона была не только посредником между чувственным миром и миром идей,
она была точным аналогом, идеальной копией реальности,
изучение которой вполне заменяет наблюдения внешнего мира.

\section{Философия и логика}
% TODO: cite this somewhere https://brickofknowledge.com/articles/programma-gilberta
Формальная логика всегда была связана с принципиальными философскими проблемами.
С одной стороны, логика выдвигает философские проблемы,
а с другой --- является средством для их решения и обсуждения.
Более того, само обоснование логики --- одна из центральных философских проблем.
С превращением формальной логики в символическую она стала применять аппарат исчислений,
а также использовать математические средства.
Это, вопреки всему, сделало связь формальной логики с философией
более глубокой и многосторонней.
Рассматривая связь логики с философией,
следует обратить внимание на роль логики в анализе глобальных философских вопросов.

Одной из таких проблем, явно показывающих связь логики и философии,
является проблема обоснования теоретического знания.
В этом плане особый интерес представляет программа, выдвинутая Д.Гильбертом. % TODO: cite
Суть его метода заключается не в формализации теории и доказательстве ее непротиворечивости,
как это обычно подчеркивается, а в обосновании вводимых <<идеальных образов>>.
Одна из задач в связи с анализом этого подхода ---
показать роль философских идей И.Канта в гильбертовской трактовке логики и математики.
Другая --- показать связь логики с обоснованием вводимых <<идеальных элементов>>
и трактовкой необходимого, математического знания.

Предполагается, что необходимым условием применения логики является,
наличие в нашем представлении внелогических объектов,
которые <<имеются в созерцании до всякого мышления>>.
Гильберт считает это той основной философской установкой,
которая обязательна как для всякого научного мышления, включая математику и физику.
При этом ключ к пониманию гильбертовской трактовки <<идеальных элементов>> математики
следует искать именно в кантовской идее \textit{схематизма чистого созерцания},
ибо это не объекты эмпирического созерцания.

% WARN: this is kinda fuzzy; bad transition
Кант отмечал <<склонность разума к расширению за узкие границы возможного опыта>>.
Там, <<где ни эмпирическое, ни чистое созерцание не держат разум в видимых рамках,
он крайне нуждается в дисциплине, которая ... удерживала бы его от крайностей и заблуждений,
так что вся философия чистого разума имеет дело только с этой негативной пользой>>.

В силу этого требуется <<совершенно особое, и при этом негативное, законодательство>>,
создающее <<систему предосторожностей и самопроверки>>. % TODO: cite
Именно этим определяется суть метода идеальных элементов Д.Гильберта.
Обязательное условие введения идеальных элементов ---
\textit{доказательство их устранимости из контекста всей теории}. % TODO: cite

В то же время велика роль идеальных элементов в построении научного знания,
в сохранении тех же математики или физики в полном объеме.
С точки зрения Гильберта совершенно неразумно выдвигать общее требование,
чтобы отдельные предложения теории допускали содержательное истолкование.
<<В теоретической физике только известная часть комбинаций и следстви
из физических законов может быть контролируема опытом, --- подобно тому,
как в моей теории доказательства только действительные высказывания
могут быть непосредственно проверяемы>>. % TODO: cite
\end{document}

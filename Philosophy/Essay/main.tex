\documentclass[a4page]{article}
\usepackage[14pt]{extsizes} % для того чтобы задать нестандартный 14-ый размер шрифта
\usepackage[utf8]{inputenc}
\usepackage[T1, T2A]{fontenc}
\usepackage[utf8]{inputenc}
\usepackage[russian]{babel} % поддержка русского языка
\usepackage{amsmath}  %  математические символы
\usepackage[left=20mm, top=15mm, right=15mm, bottom=30mm, footskip=15mm]{geometry} % настройки полей документа
\usepackage{indentfirst} % по умалчанию убирается отступ у первого абзаца в секции, это отменяет это.
\usepackage{paralist} % добавить компактные списки (compactitem, compactenum, compactdesc)

\usepackage{fancyvrb}
\usepackage{framed}
\usepackage{url}
\usepackage{csquotes}

\usepackage{float}
\floatstyle{ruled}

\usepackage[
	backend=biber,
	sorting=nyt,
	bibstyle=gost-numeric,
	citestyle=gost-numeric
]{biblatex}

\usepackage[
	bookmarks=true, colorlinks=true, unicode=true,
	urlcolor=black,linkcolor=black, anchorcolor=black,
	citecolor=black, menucolor=black, filecolor=black,
]{hyperref}

\addbibresource{sources.bib}

\renewcommand{\baselinestretch}{1.35}

\begin{document}

\begin{titlepage}
	\begin{center}
		\hfill \break
		\textbf{
			\large{РОССИЙСКИЙ УНИВЕРСИТЕТ ДРУЖБЫ НАРОДОВ}\\
			\normalsize{Факультет физико-математических и естественных наук}\\
			\normalsize{Кафедра прикладной информатики и теории вероятностей}\\
		}
		\vspace*{\fill}
		\Large{\textbf{<<Не геометр да не войдет>>, или как связаны математика, логика и философия?}}
		\\
		\underline{\textit{\normalsize{Итоговое эссе по философии}}}
		\vspace*{\fill}

	\end{center}

	\begin{flushright}
		Студент: \underline{Матюхин Григорий}\\ \vspace{0.5cm}
		Группа: \underline{НПИбд-01-21}
	\end{flushright}

	\begin{center} \textbf{МОСКВА} \\ 2023 г. \end{center}
	\thispagestyle{empty}

\end{titlepage}

\newpage

\tableofcontents

\newpage

\section{Введение}
Вопрос о взаимосвязи математики, логики и философии впервые был задан довольно давно.
Платон, Аристотель, Леонардо да Винчи, Давид Гильберт, Курт Гёдель ---
среди многих великих умов человечества, задававшихся этим вопросом,
и на чьи работы он повлиял.
Это не удивительно: ведь основу взаимодействия философии с какой-либо из наук
составляет потребность использования ее аппарата для проведения исследований.
Математика и логика, несомненно, более всего поддаются философскому анализу,
в силу их достаточной абстрактности.
Кроме того, даже не задумываясь мы можем обнаружить связь данных дисциплин,
а именно: и математика и философия являются теориями или наборами теорий ---
плодами размышлений, наборами понятий, соединенных логическими рассуждениями.

Цель данного эссе --- показать более подробно связь философии, логики и математики
как на протяжении истории человечества, так и в отдельных,
на первый взгляд не имеющих никакого отношения к философии работах,
а также постараться объяснить причину этой связи.

\section{Философия и математика}
История развития науки и культуры в целом свидетельствует о том,
что философия и математика, как определенные системы знаний,
возникают примерно в одно время.
С этого момента у них обнаруживается множество точек соприкосновения:
высокий уровень обобщений, единое и многое, конечное и бесконечное,
количество и качество, проблема существования и проблема истины,
противоречие и непротиворечивость, и многое другое.
Для краткости данное эссе ограничивается примерами связи математики и философии только в античное время,
хотя эта связь несомненно становилась все теснее и многограннее с каждом столетием под влиянием таких умов,
как Р.Декарт, Г.Лейбниц, И.Кант и многих других.

\subsection{Милетская, пифагорейская и атомистическая школы философии}
Взаимосвязь математики и философии видна уже в милетской школе философии.
Она проявляется, во-первых, в том, что в математике переходят к доказательствам,
в которых нуждалась и философия;
а во-вторых, в том, что первые милетские философы в математике
<<выходят на путь абстрактных обобщающих построений>> \cite{birth-development-of-ideas},
которые превращают математику во всеобще знание.

Математика начинает оказывать еще более сильное влияние на философию в пифагорейской школе,
философы которой обнаружили в математике выражение глубинной сущности мира,
нечто связанное с истинной и неизменной природой вещей.
Основной тезис пифагореизма состоит в том, что <<все есть число>>.

Как подражание числам в пифагорейской школе стали рассматриваться
не только чувственно воспринимаемые вещи, но и их свойства и отношения.
По аналогии со свойствами того или иного числа или числового соотношения,
их стали трактовать как проявление числовой гармонии.
Можно сказать, что пифагореизм стал первой философской теорией математики.

Последователи атомистического учения Левкиппа и Демокрита
также испытали на себе огромнейшее влияние математики.
В их учении, в противовес пифагорейцам,
геометрические фигуры считались не умозрительными сущностями,
а материальными телами, состоящими из атомов.
Физическое здесь понималось как логически предшествующее математическому,
умозрительному, абстрактному,
а математические закономерности выступали как вторичные по отношению к атомам.
Атомы же считались и материальной причиной вещей, причиной их существования,
и первыми сущностями, невидимыми простым глазом, а <<зримые>> лишь умом.
В определенной мере можно сказать, что атомизм предвосхитил идеи математического естествознания.

Созданная Демокритом концепция математики, называемая кониепцией математического атомизма,
дала ему возможность решить проблему правомерности теоретических построений математики,
не обращаясь к чувственным образам, как это было у Протагора.

\subsection{Платон и Аристотель}
В дальнейшем близость философии и математики обнаруживается у Платона и Аристотеля.

У Платона математика и философия, так же как и у пифагорейцев,
связываются между собой, прежде всего, посредством числа.
В диалоге «Государство» Платон утверждает, что сущность вещи,
скрывающаяся за ее кажимостью, есть математическое.
К истине у Платона ведут именно арифметика и счет, то есть наука о числе.
<<Эта наука, --- отмечает Платон, --- подходит для того,
чтобы установить закон и убедить всех,
кто собирается занять высокие должности в государстве, обратиться к искусству счета>>
и придти <<с помощью самого мышления к созерцанию природы чисел>>
для облегчения <<самой душе ее обращение от становления к истинному бытию>> \cite{plato-compilation}.

Платон пошел дальше пифагорейцев в исследованиях реального физического мира,
построенного, с его точки зрения, на основе математических идей.
Этот идеальный мир и был для него истинной реальностью,
к познанию которой он стремился с помощью математики.
Математика для Платона была не только посредником между чувственным миром и миром идей,
она была точным аналогом, идеальной копией реальности,
изучение которой вполне заменяет наблюдения внешнего мира.

Мысль о том, что с помощью математики вполне возможно изучение физической реальности,
получая о нем истинное знание, проводит в своей философии и Аристотель.
Он солидарен с Платоном в том, что в основе мироздания лежит некая математическая идея,
некий математический план построения внешнего мира.
Однако в противовес Платону он считает,
что <<математические вещи, вопреки словам некоторых философов,
не существуют отдельно от чувственных вещей и не бывают их причинами>> \cite{aristotle-metaphysics}.

Еще одной проблемой, сближающей философию и математику,
которой занимался Аристотель, была проблема бесконечного.
С точки зрения Аристотеля бесконечность --- это <<то, вне чего всегда есть что-нибудь>>.
Она не преeставляет собой какую-то определенную сущность, у нее нет начала, но нет и конца.
Бесконечное --- это не ставшее, но становящееся.
Иначе говоря, бесконечность для Аристотеля может рассматриваться лишь как потенциальная бесконечность;
актуальной бесконечности, как бесконечности ставшей, завершенной,
с его точки зрения, не существует, ибо все, что имеет предел не может считаться бесконечным.

\subsection{Современность}
Анализ уже современного этапа развития познания убедительно показывает,
что философия активно использует математический аппарат для выявления
и исследования закономерностей в тех или иных областях действительности.
Примерами могут служить анализ языковых явлений или
построение моделей диалектики и других философских систем.
Кроме того, философы в вполне осознанно используют
достижения математического познания для философского анализа,
а возникающие внутри математики трудности и проблемы ---
для осмысления общих проблем научного познания.
Математика, в свою очередь, обращается к философии,
когда для решения каких-либо внутренних проблем ей не достает своих собственных средств,
когда возникает необходимость выйти за пределы математической науки
(например, проблемы генезиса и обоснования математики,
существования математических объектов,
границ применения математического знания и возможностей методов и т. д.).
А нередко, решая сугубо математические задачи,
математики неосознанно опираются на те или иные философские,
мировоззренческие установки и принципы, неявно используют методы, разработанные философией.

Конечно, совершенно ясно,
что подходы к решению той или иной проблемы в философии и математике отличаются друг от друга.
Тем не менее, взаимовлияние математики и философии и сегодня в отдельных случаях
весьма сильно и продуктивно, если не для обеих, то, по крайней мере, для какой-то одной из них.

\section{Философия и логика}
Формальная логика всегда была связана с философией и с философскими проблемами.
Она выдвигает философские проблемы, а так же является средством для их решения и обсуждения.
Более того, само обоснование логики --- одна из центральных философских проблем.
С превращением формальной логики в символическую она стала применять математический аппарат исчислений,
что, как может показаться на первый взгляд, отдаляет ее от философии.
Но, как мы уже выяснили в предыдущей части,
математика и философия неразрывно связаны между собой с самого их зарождения.
Поэтому применение математических средств в логике
только сделало ее связь с философией еще более глубокой и многосторонней.
Рассматривая эту связь,
следует обратить внимание на роль логики в анализе глобальных гносеологических философских вопросов.

\subsection{Программа Д.Гилберта и ее связь с схематизмом чистого созерцания И.Канта}
Одним из таких вопросов, является проблема обоснования теоретического знания.
В этом плане особый интерес представляет программа Давида Гильберта \cite{sep-hilbert-program}.
Эта программа призвана формализовать всю математику в аксиоматическом виде,
а затем доказать, что полученная аксиоматизация непротиворечива.
Одной из важнейших частей программы является обоснование <<идеальной математики>> или <<идеальных элементов>>,
являющихся началом всех <<реальных>> математических предложений.
Анализируя этот подход, необходимо показать роль философских идей Иманнуила Канта
в гильбертовской трактовке логики и математики,
а также показать связь логики с обоснованием вводимых <<идеальных элементов>>
и трактовкой теориетического математического знания.

Идея Гилберта заключается в том, что необходимым условием применения логики является
наличие неких внелогических объектов, которые <<имеются в созерцании до всякого мышления>>.
По его мнению такого рода объекты обязательны для всякого научного мышления, включая математику и физику.
Примером необходимости таких объектов служит тот факт,
что даже элементарная математика <<уже не останавливается на точке зрения наглядной теории чисел>>
и вместе с действительными включает идеальные высказывания, предполагающие введение объектов,
которые в принципе не могут существовать ни в эмпирическом, ни в чистом созерцании
\cite{the-foundations-of-geometry}.
Эти объекты имеют сильное сходство с \textit{схематизмом чистого созерцания} Канта.
Именно в его \textit{трансцендентальной аналитике} и следует искать ключ
к пониманию гильбертовских <<реальной>> и <<идельной>> математик.

Кант отмечал <<склонность разума к расширению за узкие границы возможного опыта>>.
Там, <<где ни эмпирическое, ни чистое созерцание не держат разум в видимых рамках ...
он крайне нуждается в дисциплине, которая ... удерживала бы его от крайностей и заблуждений,
так что вся философия чистого разума имеет дело только с этой негативной пользой>>
\cite{iphas-kant-pure-reason}.
В силу этого требуется <<совершенно особое, и при этом негативное, законодательство>>,
создающее <<систему предосторожностей и самопроверки>>.
Именно этим определяется суть метода идеальных элементов Д.Гильберта.
Обязательное условие введения идеальных элементов ---
\textit{доказательство их устранимости из контекста всей теории}.
Идеальные элементы можно вводить в теорию лишь в том случае, если те соотношения,
которые выявляются в ней после расширения для прежних объектов при исключении идеальных образов,
верны и в старой области. Иначе говоря, такие объекты вводятся лишь для простоты,
удобства и единообразия применяемых методов \cite{logic-and-philosophy}.

Роль идеальных элементов так же велика в построении научного знания,
в сохранении тех же математики или физики в полном объеме и единообразии.
С точки зрения Гильберта совершенно неразумно выдвигать общее требование,
чтобы отдельные положения теории допускали содержательное истолкование.
<<В теоретической физике только известная часть комбинаций и следствий
из физических законов может быть контролируема опытом, --- подобно тому,
как в моей теории доказательства только действительные высказывания
могут быть непосредственно проверяемы.
Ценность чистого доказательсва существования в том иммено и состоит,
что благодаря ему исключаются отдельные построения и многи разнообразные постоения объединяются одной идеей,
вследствие чего четко выступает только то, что существенно для доказательсва.>>\cite{the-foundations-of-geometry}.

Как можно заметить, применение точных методов к логике,
сближение логики с математикой поднимает новый пласт важнейших философских,
теоретико-познавательных вопросов.
Хотя нам кажется странным, когда об абстрактных сущностях и связях
говорят как о формальных, оторванных от познавательной деятельности.
На самом деле создание системы абстрактных,
идеальных конструктов является важнейшим аспектом, самой сутью познавательной деятельности.

\section{Вывод}
В заключение хотелось бы подчеркнуть, что абстрактные, обобщенные понятия,
используемые в математике и логике тесно переплетены с глубокими философскими устремлениями,
с размышлениями о сущностях физических и метафизических объектов, сущности самой реальности,
принципах познавательной и теориетической деятельностях
проливают подчас неожиданный свет на эти философские вопросы.

\newpage
\addcontentsline{toc}{section}{Список литературы}
\printbibliography

\end{document}

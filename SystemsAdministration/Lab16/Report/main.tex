\documentclass[12pt]{article}
\usepackage[T1, T2A]{fontenc}
\usepackage[utf8]{inputenc}
\usepackage[russian]{babel}
\usepackage{hyperref}
\usepackage{graphicx}
\graphicspath{ {../Images/} }

\author{Григорий Матюхин}
\date{\today}
\title{Лабораторная работа \textnumero16.\\Программный RAID}

\begin{document}
\maketitle
\newpage
\tableofcontents
\newpage
\section{Цель работы}
Освоить работу с RAID-массивами при помощи утилиты \texttt{mdadm}.

\section{Последовательность выполнения работы}
\subsection{Создание RAID-диска}
\begin{enumerate}
	1. Запустите виртуальную машину. Получите полномочия администратора:
	2. Проверьте наличие созданных вами ранее дополнительных дисков:
	3. Создайте на каждом из дисков раздел:
	5. Просмотрите, какие типы партиций, относящиеся к RAID, можно задать:
	7. Просмотрите состояние дисков:
	9. При помощи утилиты \texttt{mdadm} создайте массив RAID 1 из двух дисков:
	10. Проверьте состояние массива RAID, используя команды
	11. Создайте файловую систему на RAID:
	12. Подмонтируйте RAID:
	13. Для автомонтирования добавьте запись в \texttt{/etc/fstab}:
	14. Сымитируйте сбой одного из дисков:
	15. Удалите сбойный диск:
	16. Замените диск в массиве:
	17. Посмотрите состояние массива и опишите его в отчёте.
	18. Удалите массив и очистите метаданные:
\end{enumerate}

\subsection{RAID-массив с горячим резервом (hotspare)}
\begin{enumerate}
	2. Создайте массив RAID 1 из двух дисков:
	3. Добавьте третий диск:
	4. Подмонтируйте \texttt{/dev/md0}:
	5. Проверьте состояние массива:
	6. Сымитируйте сбой одного из дисков:
	7. Проверьте состояние массива:
	8. Удалите массив и очистите метаданные:
\end{enumerate}

\subsection{Преобразование массива RAID 1 в RAID 5}
\begin{enumerate}
	2. Создайте массив RAID 1 из двух дисков:
	3. Добавьте третий диск:
	4. Подмонтируйте \texttt{/dev/md0}:
	5. Проверьте состояние массива:
	6. Измените тип массива RAID:
	7. Проверьте состояние массива:
	8. Измените количество дисков в массиве RAID 5:
	9. Проверьте состояние массива:
	10. Удалите массив и очистите метаданные:
	11. Закомментируйте запись в \texttt{/etc/fstab}:
\end{enumerate}

\section{Контрольные вопросы}
\begin{enumerate}
	1. Приведите определение RAID.
	2. Какие типы RAID-массивов существуют на сегодняшний день?
	3. Охарактеризуйте RAID 0, RAID 1, RAID 5, RAID 6, опишите алгоритм работы,
	назначение, приведите примеры применения
\end{enumerate}

\section{Вывод}
В ходе выполнения данной работы я освоил работу с RAID-массивами при помощи утилиты \texttt{mdadm}.

\end{document}
